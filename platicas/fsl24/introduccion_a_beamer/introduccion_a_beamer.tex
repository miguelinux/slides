% ex: ts=2 sw=2 sts=2 et filetype=tex
% SPDX-License-Identifier: CC-BY-SA-4.0

\documentclass[aspectratio=169]{beamer}

\usetheme{Boadilla}

\definecolor{fslcolor}{rgb}{0.0, 0.564, 0.549} % #00908C
\usecolortheme[named=fslcolor]{structure}
%\usepackage{../../../styles/beamercolorthemedracula}

\setbeamertemplate{navigation symbols}{}

\graphicspath{{../../img/}{../../../latex/img/}}
\usepackage[utf8]{inputenc}
\usepackage[T1]{fontenc}
\usepackage{graphicx}
\usepackage{listings}
%\usepackage{colortbl}
\usepackage{tikz}
%\usetikzlibrary{trees}

\title[\LaTeX, Beamer y Git]{Haz tus presentaciones con \LaTeX,
Beamer y Git}

\author[\textcircled{cc} BY-SA 4.0]{Miguel Bernal Marin}

\institute[FSL 2024]
{
Festival de Software Libre Vallarta 2024\\
\medskip
\textit{\href{mailto:miguel.bernal.marin@gmail.com}{miguel.bernal.marin@gmail.com}}\\
Telegram: \textit{\href{https://t.me/miguelinux}{@miguelinux}}
}
\date{
  4 de mayo de 2024
}

% ex: ts=2 sw=2 sts=2 et filetype=tex
% SPDX-License-Identifier: CC-BY-SA-4.0

%New colors defined below
\definecolor{editorNumeros}{rgb}{0.5,0.5,0.5}
\definecolor{editorGray}{rgb}{0.95, 0.95, 0.95}
\definecolor{editorOcher}{rgb}{1, 0.5, 0} % #FF7F00 -> rgb(239, 169, 0)
\definecolor{editorGreen}{rgb}{0, 0.5, 0} % #007C00 -> rgb(0, 124, 0)
\definecolor{editorBrown}{rgb}{0.69,0.31,0.31}

%Code listing style named "griscmd"
\lstdefinestyle{griscmd}{
  % General design
  backgroundcolor=\color{editorGray},
  basicstyle={\ttfamily},
  numberstyle=\tiny\color{editorNumeros},
  xleftmargin={0.2cm},
  numbersep=5pt,
  numbers=left,
  stepnumber=1,
  firstnumber=1,
  numberfirstline=true,
  % Code design
  identifierstyle=\color{black},
  keywordstyle=\color{blue}\bfseries,
  ndkeywordstyle=\color{editorGreen}\bfseries,
  stringstyle=\color{editorOcher}\ttfamily,
  commentstyle=\color{editorBrown}\ttfamily,
  % Code
  alsodigit={.:;},
  tabsize=2,
  showtabs=false,
  showspaces=false,
  showstringspaces=false,
  extendedchars=true,
  breaklines=true,
  % Español
  literate=%
  {á}{{\'a}}1 {é}{{\'e}}1 {í}{{\'i}}1 {ó}{{\'o}}1 {ú}{{\'u}}1
  {Á}{{\'A}}1 {É}{{\'E}}1 {Í}{{\'I}}1 {Ó}{{\'O}}1 {Ú}{{\'U}}1
  {¡}{{!`}}1  {¿}{{?`}}1
  {ñ}{{\~n}}1 {Ñ}{{\~N}}1
}

%"griscmd" code listing set
\lstset{style=griscmd}


%% ex: ts=2 sw=2 sts=2 et filetype=tex
% SPDX-License-Identifier: CC-BY-SA-4.0

\definecolor{codeBakground}{rgb}{0.95,0.95,0.95}
\definecolor{codeComment}{RGB}{169, 169, 169}
\definecolor{codeKeyword}{RGB}{127, 0, 85}
\definecolor{codeNumbers}{rgb}{0.5,0.5,0.5}
\definecolor{codeString}{RGB}{0, 100, 0}
\definecolor{light-gray}{gray}{0.90}

%Code listing style named "mystyle"

\lstdefinestyle{mystyle} {%
  % General design
  backgroundcolor=\color{codeBakground},
  basicstyle=\small,
  numberstyle=\tiny\color{codeNumbers},
  xleftmargin={0.2cm},
  numbersep=5pt,
  numbers=left,
  stepnumber=1,
  firstnumber=1,
  numberfirstline=true,
  % Code design
  identifierstyle=\color{black},
  keywordstyle=\color{codeKeyword}\bfseries,
  keywordstyle={[2]\color{codeKeyword}},
  stringstyle=\color{codeString}\ttfamily,
  commentstyle=\color{codeComment}\ttfamily,
  % Code
  tabsize=2,
  showtabs=false,
  showspaces=false,
  showstringspaces=false,
  extendedchars=true,
  breaklines=true,
  breakatwhitespace=false,
  captionpos=b,
  keepspaces=true,
  upquote=true,                      % requires textcomp
  % Español
  literate=%
  {á}{{\'a}}1 {é}{{\'e}}1 {í}{{\'i}}1 {ó}{{\'o}}1 {ú}{{\'u}}1
  {Á}{{\'A}}1 {É}{{\'E}}1 {Í}{{\'I}}1 {Ó}{{\'O}}1 {Ú}{{\'U}}1
  {¡}{{!`}}1  {¿}{{?`}}1
  {ñ}{{\~n}}1 {Ñ}{{\~N}}1
}

%"mystyle" code listing set
\lstset{style=mystyle}



\logo{\includegraphics[height=1cm]{fsl.png}}

\titlegraphic{
    \includegraphics[width=3cm]{fsl.png}
}

%------------------------------------------------------------
%The next block of commands puts the table of contents at the
%beginning of each section and highlights the current section:
\AtBeginSection[]
{
  \begin{frame}
    \frametitle{Contenido}
    \tableofcontents[currentsection]
  \end{frame}
}
%------------------------------------------------------------

\newcommand{\nologo}{\setbeamertemplate{logo}{}} % command to set the logo to nothing
\newcommand{\pausa}{\pause} % Para usar una pausa en las presentaciones
%\newcommand{\pausa}{}      % Para que NO salgan las pausas

\begin{document}

{
\nologo
\begin{frame}
    \titlepage
\end{frame}
}

% ex: ts=2 sw=2 sts=2 et filetype=tex
% SPDX-License-Identifier: CC-BY-SA-4.0

\begin{frame}[c]{¡Hola!}

  \begin{columns}
    \column{0.7\textwidth}
      Miguel Bernal Marin

      \vspace{\baselineskip}
      OS Developer Engineer \textbf{@Intel}
    \column{0.3\textwidth}
      \includegraphics[scale=0.2]{miguelinux-color.png}
  \end{columns}

  Profesor:
  \begin{itemize}
    \item \href{https://tsj.mx/}{Instituto Tecnológico Superior
      de Jalisco (Zapopan)}
    \item \href{https://iteso.mx/}{ITESO}
  \end{itemize}

  \vspace{\baselineskip}
  %\begin{center}
    \href{https://t.me/linuxeroszapopan}{\includegraphics[height=1.8cm]{lnxzpn.png}}
    \hspace{1cm}
    \href{https://tecmm.edu.mx/}{\includegraphics[height=1.8cm]{tsj.png}}
    \hspace{1cm}
    \href{https://iteso.mx/}{\includegraphics[height=1.8cm]{iteso-logo.png}}
  %\end{center}
\end{frame}

\input{../../common/intel-disclaimer.tex}

\begin{frame}
    \frametitle{Contenido}
    \tableofcontents
\end{frame}

% ex: ts=2 sw=2 sts=2 et filetype=tex
% SPDX-License-Identifier: CC-BY-SA-4.0

\section{\LaTeX{} y Beamer}

\begin{frame}[c]{¿Qué es \LaTeX{}?}
  \LaTeX{} (escrito LaTeX en texto sin formato) es un sistema de composición
  de textos orientado a la \textbf{creación de documentos} escritos que
  presenten una alta calidad tipográfica.

  \vspace{\baselineskip}
  Por sus características y posibilidades, se usa de forma especialmente
  intensa en la generación de \textbf{artículos y libros científicos} que
  incluyen, entre otros elementos, \textbf{expresiones matemáticas}.
\end{frame}

\begin{frame}[c]{¿Qué es Beamer?}
  \textbf{Beamer} es una clase de LaTeX para la creación de presentaciones.

  \vspace{\baselineskip}
  Funciona con \textbf{pdflatex}, \textbf{dvips} y \textbf{LyX}.
  El nombre viene del vocablo alemán "beamer", un pseudo-anglicismo que
  significa videoproyector.

  \vspace{\baselineskip}
  Al estar basado en LaTeX, Beamer es especialmente útil para preparar
  presentaciones en las que es necesario mostrar gran cantidad de expresiones
  matemáticas, el fuerte de dicho sistema de maquetación.
\end{frame}

\begin{frame}[c]{Lista de códigos}
  \textbf{listados (listings)} es usado para componer listados de código
  fuente usando LaTeX

  \vspace{\baselineskip}
  El paquete permite al usuario componer programas (código de programación)
  dentro de LaTeX; TeX lee el código fuente directamente; no se necesita
  ningún procesador frontal.

  \vspace{\baselineskip}
  Las palabras clave, los comentarios y las cadenas se pueden componer
  usando diferentes estilos.
\end{frame}

\section{\LaTeX{} y Git}

\begin{frame}[c]{Git}
  \textbf{Git} es un software de control de versiones diseñado por
  \texttt{Linus Torvalds},
  pensando en la eficiencia, la confiabilidad y compatibilidad del
  mantenimiento de versiones de aplicaciones cuando estas tienen un
  gran número de archivos de código fuente.

  \vspace{\baselineskip}
  Su propósito es llevar registro de los cambios en archivos de computadora
  incluyendo coordinar el trabajo que varias personas realizan sobre archivos
  compartidos en un repositorio de código.
\end{frame}

\begin{frame}[fragile]
  \frametitle{Ejemplo de Beamer}
  \begin{lstlisting}[language={[LaTeX]TeX}]
\documentclass{beamer}
\title{Titulo}
\author{Anónimo}
\institute{FSL}
\date{2021}

\begin{document}

\frame{\titlepage}

\begin{frame}[c]{Titulo}
  Texto
\end{frame }

\end{document}
  \end{lstlisting}
\end{frame}

\begin{frame}[fragile]
  \frametitle{Divide y venceras}
  Separar el contenido de la portada/configuración de la presentación.
  \begin{lstlisting}[language=Bash,numbers=none]
$ tree ssh
ssh/
|-- contenido.tex
|-- ssh.tex
$
  \end{lstlisting}
\end{frame}

\begin{frame}[fragile]
  \frametitle{Divide y venceras}
  Crea diferentes branches para cada una de las versiones/estilos de las
  presentaciones.
  \begin{lstlisting}[language=Bash,numbers=none]
$ git branch
estilo1
estilo2
main
oscuro1
oscuro2
$
  \end{lstlisting}
\end{frame}

\begin{frame}[c]{Usando diferentes estilos}
  \includegraphics[height=3.8cm]{s1v1.png}
  \includegraphics[height=3.8cm]{s1v2.png}
  \includegraphics[height=3.8cm]{s1v3.png}
  \includegraphics[height=3.8cm]{s1v4.png}
\end{frame}

\begin{frame}[c]{Git Large File Storage (LFS)}
  LFS es una extensión Git de código abierto para versionar archivos
  grandes.

  \vspace{\baselineskip}
  Git Large File Storage (LFS) reemplaza archivos grandes, como muestras
  de audio, videos, conjuntos de datos y gráficos con punteros de texto
  dentro de Git, mientras almacena el contenido del archivo en un servidor
  remoto como GitHub.com o GitHub Enterprise.
\end{frame}

\begin{frame}[fragile]
  \frametitle{Git LFS}
  El archivo de atributos para el git LFS es:
  \begin{lstlisting}[language=Bash,numbers=none]
$ cat .gitattributes
*.png filter=lfs diff=lfs merge=lfs -text
*.jpg filter=lfs diff=lfs merge=lfs -text
$
  \end{lstlisting}
\end{frame}

\begin{frame}[c]{}
  \pausa
  \vspace{\baselineskip}
\end{frame}



\begin{frame}{Agradecimientos y créditos}

  \begin{center}
    \huge ¡Gracias!
  \end{center}

  El código fuente de esta presentación la puedes encontrar en
  \begin{itemize}
    \item \href{https://github.com/miguelinux/slides}
      {https://github.com/miguelinux/slides}
      \{platicas/fsl24/introduccion\_a\_beamer\}

  \end{itemize}


  \vspace{\baselineskip}
  El material está basado en
  \begin{itemize}
    \item
      \href{https://es.wikipedia.org/wiki/Wikipedia:Portada}
      {https://es.wikipedia.org/}
  \end{itemize}
\end{frame}

\end{document}
