% ex: ts=2 sw=2 sts=2 et filetype=tex
% SPDX-License-Identifier: CC-BY-SA-4.0

\documentclass[aspectratio=169]{beamer}

\usetheme{Boadilla}

\definecolor{kcdcolor}{rgb}{0.255, 0.412, 0.882}
\usecolortheme[named=kcdcolor]{structure}

\setbeamertemplate{navigation symbols}{}

\graphicspath{{../img/}{../../linux/img/}}
\usepackage[utf8]{inputenc}
\usepackage[T1]{fontenc}
\usepackage{graphicx}
\usepackage{listings}
%\usepackage{colortbl}
\usepackage{tikz}
%\usetikzlibrary{trees}

\title[Quadlets de Podman]{Como usar los Quadlets de Podman}

\author[\textcircled{cc} BY-SA 4.0]{Miguel Bernal Marin}

\institute[KCD 2025]
{
Kubernetes Community Day Guadalajara \\
\medskip
\textit{\href{mailto:miguel.bernal.marin@gmail.com}{miguel.bernal.marin@gmail.com}}\\
Telegram: \textit{\href{https://t.me/miguelinux}{@miguelinux}}
}
\date{
  29 de marzo de 2025
}

\input{../../common/lstGris}

\logo{\includegraphics[height=2cm]{img/kcd.png}}

%------------------------------------------------------------
%The next block of commands puts the table of contents at the
%beginning of each section and highlights the current section:
\AtBeginSection[]
{
{
\nologo
\usebackgroundtemplate{\includegraphics[width=\paperwidth]{img/kcd-agenda.png}}
  \begin{frame}
    \frametitle{Contenido}
    \tableofcontents[currentsection]
  \end{frame}
}
}
%------------------------------------------------------------

\newcommand{\nologo}{\setbeamertemplate{logo}{}} % command to set the logo to nothing
\newcommand{\pausa}{\pause} % Para usar una pausa en las presentaciones
%\newcommand{\pausa}{}      % Para que NO salgan las pausas

\begin{document}

{
\nologo
\setbeamertemplate{footline}{}
\usebackgroundtemplate{\includegraphics[width=\paperwidth]{img/kcd-titulo.png}}
\begin{frame}
    \titlepage
\end{frame}
}

% ex: ts=2 sw=2 sts=2 et filetype=tex
% SPDX-License-Identifier: CC-BY-SA-4.0

\begin{frame}[c]{¡Hola!}

  \begin{columns}
    \column{0.7\textwidth}
      Miguel Bernal Marin

      \vspace{\baselineskip}

    \column{0.3\textwidth}
      \includegraphics[scale=0.2]{miguelinux-color.png}
  \end{columns}

  Profesor:
  \begin{itemize}
    \item \href{https://tsj.mx/}{Instituto Tecnológico Superior
      de Jalisco (Zapopan)}
    \item \href{https://iteso.mx/}{ITESO}
  \end{itemize}

  \vspace{\baselineskip}
  %\begin{center}
    \href{https://t.me/linuxeroszapopan}{\includegraphics[height=1.8cm]{lnxzpn.png}}
    \hspace{1cm}
    \href{https://tecmm.edu.mx/}{\includegraphics[height=1.8cm]{tsj.png}}
    \hspace{1cm}
    \href{https://iteso.mx/}{\includegraphics[height=1.8cm]{iteso-logo.png}}
  %\end{center}
\end{frame}

% ex: ts=2 sw=2 sts=2 et filetype=tex
% SPDX-License-Identifier: CC-BY-SA-4.0

\begin{frame}[c]{Legal Disclaimer}
  This presentation is for informational purposes only.
  INTEL MAKES NO WARRANTIES, EXPRESS OR IMPLIED, IN THIS SUMMARY.

  \vspace{\baselineskip}
  Intel and the Intel logo are trademarks of Intel Corporation in the
  U.S. and/or other countries.

  \vspace{\baselineskip}
  * Other names and brands an logos may be claimed as the property of others.

  \vspace{\baselineskip}
  Copyright © 2025, Intel Corporation. All rights reserved.
\end{frame}


{
\nologo
\setbeamertemplate{footline}{}
\usebackgroundtemplate{\includegraphics[width=\paperwidth]{img/kcd-agenda.png}}
\begin{frame}
    \frametitle{Contenido}
    \tableofcontents
\end{frame}
}

% ex: ts=2 sw=2 sts=2 et filetype=tex
% SPDX-License-Identifier: CC-BY-SA-4.0

\section{¿Qué son los contenedores?}



\begin{frame}{Agradecimientos y créditos}

  \begin{center}
    \huge ¡Gracias!
  \end{center}

  El código fuente de esta presentación la puedes encontrar en
  \begin{itemize}
    \item \href{https://github.com/miguelinux/slides}
      {https://github.com/miguelinux/slides}
      \{platicas/fsl24/introduccion\_a\_beamer\}

  \end{itemize}


  \vspace{\baselineskip}
  El material está basado en
  \begin{itemize}
    \item
      \href{https://es.wikipedia.org/wiki/Wikipedia:Portada}
      {https://es.wikipedia.org/}
  \end{itemize}
\end{frame}

\end{document}
