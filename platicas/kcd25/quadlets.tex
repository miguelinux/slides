% ex: ts=2 sw=2 sts=2 et filetype=tex
% SPDX-License-Identifier: CC-BY-SA-4.0

\documentclass[aspectratio=169]{beamer}

\usetheme{Boadilla}

\definecolor{kcdcolor}{rgb}{0.255, 0.412, 0.882}
\usecolortheme[named=kcdcolor]{structure}

\setbeamertemplate{navigation symbols}{}

\graphicspath{{../img/}{../../linux/img/}}
\usepackage[utf8]{inputenc}
\usepackage[T1]{fontenc}
\usepackage{graphicx}
\usepackage{listings}
%\usepackage{colortbl}
\usepackage{tikz}
%\usetikzlibrary{trees}

\title[Quadlets de Podman]{Como usar los Quadlets de Podman}

\author[\textcircled{cc} BY-SA 4.0]{Miguel Bernal Marin}

\institute[KCD 2025]
{
Kubernetes Community Day Guadalajara \\
\medskip
\textit{\href{mailto:miguel.bernal.marin@gmail.com}{miguel.bernal.marin@gmail.com}}\\
Telegram: \textit{\href{https://t.me/miguelinux}{@miguelinux}}
}
\date{
  29 de marzo de 2025
}

\input{../../common/lstGris}
\lstset{language=Bash}

\logo{\includegraphics[height=2cm]{img/kcd.png}}

%------------------------------------------------------------
%The next block of commands puts the table of contents at the
%beginning of each section and highlights the current section:
\AtBeginSection[]
{
{
\nologo
\usebackgroundtemplate{\includegraphics[width=\paperwidth]{img/kcd-agenda.png}}
  \begin{frame}
    \frametitle{Contenido}
    \tableofcontents[currentsection]
  \end{frame}
}
}
%------------------------------------------------------------

\newcommand{\nologo}{\setbeamertemplate{logo}{}} % command to set the logo to nothing
\newcommand{\pausa}{\pause} % Para usar una pausa en las presentaciones
%\newcommand{\pausa}{}      % Para que NO salgan las pausas

\begin{document}

{
\nologo
\setbeamertemplate{footline}{}
\usebackgroundtemplate{\includegraphics[width=\paperwidth]{img/kcd-titulo.png}}
\begin{frame}
    \titlepage
\end{frame}
}

% ex: ts=2 sw=2 sts=2 et filetype=tex
% SPDX-License-Identifier: CC-BY-SA-4.0

\begin{frame}[c]{¡Hola!}

  \begin{columns}
    \column{0.7\textwidth}
      Miguel Bernal Marin

      \vspace{\baselineskip}

    \column{0.3\textwidth}
      \includegraphics[scale=0.2]{miguelinux-color.png}
  \end{columns}

  Profesor:
  \begin{itemize}
    \item \href{https://tsj.mx/}{Instituto Tecnológico Superior
      de Jalisco (Zapopan)}
    \item \href{https://iteso.mx/}{ITESO}
  \end{itemize}

  \vspace{\baselineskip}
  %\begin{center}
    \href{https://t.me/linuxeroszapopan}{\includegraphics[height=1.8cm]{lnxzpn.png}}
    \hspace{1cm}
    \href{https://tecmm.edu.mx/}{\includegraphics[height=1.8cm]{tsj.png}}
    \hspace{1cm}
    \href{https://iteso.mx/}{\includegraphics[height=1.8cm]{iteso-logo.png}}
  %\end{center}
\end{frame}

% ex: ts=2 sw=2 sts=2 et filetype=tex
% SPDX-License-Identifier: CC-BY-SA-4.0

\begin{frame}[c]{Legal Disclaimer}
  This presentation is for informational purposes only.
  INTEL MAKES NO WARRANTIES, EXPRESS OR IMPLIED, IN THIS SUMMARY.

  \vspace{\baselineskip}
  Intel and the Intel logo are trademarks of Intel Corporation in the
  U.S. and/or other countries.

  \vspace{\baselineskip}
  * Other names and brands an logos may be claimed as the property of others.

  \vspace{\baselineskip}
  Copyright © 2025, Intel Corporation. All rights reserved.
\end{frame}


{
\nologo
\usebackgroundtemplate{\includegraphics[width=\paperwidth]{img/kcd-agenda.png}}
\begin{frame}
    \frametitle{Contenido}
    \tableofcontents
\end{frame}
}

% ex: ts=2 sw=2 sts=2 et filetype=tex
% SPDX-License-Identifier: CC-BY-SA-4.0

\section{Podman y Quadlets}

\begin{frame}[c]{Podman (Pod Manager)}
    \begin{columns}
        \column{0.4\textwidth}
        \begin{center}
            \includegraphics[scale=0.4]{lfs/podman.png}
        \end{center}
        \column{0.6\textwidth}
        \begin{itemize}
          \item Es una herramienta sencilla y \textbf{sin demonio}.
          \pausa
          \item Un \textbf{motor de contenedores} con todas las funciones.
          \pausa
          \item Proporciona una línea de comandos comparable a
            \textbf{Docker-CLI}
            que facilita la transición desde otros motores de contenedores
            y permite la gestión de \textbf{pods, contenedores e imágenes}.
        \end{itemize}
    \end{columns}
\end{frame}

\begin{frame}[c]{Quadlets}
  \begin{itemize}
    \item Está disponible a partir de Podman v4.4.
    \pausa
    \item Se \textbf{describe} cómo ejecutar un contenedor en un formato
      muy similar al de los archivos de unidad de \textbf{systemd}.
    \pausa
    \item Las descripciones del contenedor se centran en los
      \textbf{detalles relevantes} del contenedor y \underline{ocultan}
      los detalles técnicos de su ejecución en \textbf{systemd}.
  \end{itemize}
\end{frame}

\begin{frame}[c]{Directorios}

  Los archivos de quadlets para el usuario \textbf{root} se pueden
  colocar en los siguientes directorios, ordenados por prioridad.

  Esto significa que los quadlets con nombre duplicado que se
  encuentren en \textbf{/run} tienen prioridad sobre los de
  \textbf{/etc} y \textbf{/usr}:
  \pausa
  \begin{alertblock}{Cuadlets temporales, generalmente usados para pruebas:}
    \begin{itemize}
      \item /run/containers/systemd/
    \end{itemize}
  \end{alertblock}
  \pausa
  \begin{exampleblock}{Cuadlets definidos por el administrador del sistema:}
    \begin{itemize}
      \item /etc/containers/systemd/
    \end{itemize}
  \end{exampleblock}
  \pausa
  \begin{block}{Cuadlets definidos por la distribución:}
    \begin{itemize}
      \item /usr/share/containers/systemd/
    \end{itemize}
  \end{block}
\end{frame}

\begin{frame}[c]{Directorios}

  Los archivos de quadlets para los usuario \textbf{no-root} se pueden
  colocar en los siguientes directorios:

  \vspace{\baselineskip}
  \begin{exampleblock}{Usuario no-root:}
    \begin{itemize}
      \item \$XDG\_RUNTIME\_DIR/containers/systemd/
      \pausa
      \item \$XDG\_CONFIG\_HOME/containers/systemd/
      \pausa
        \begin{itemize}
          \item \$HOME/.config/containers/systemd/
        \end{itemize}
      \pausa
      \item /etc/containers/systemd/users/\$(UID)
      \pausa
      \item /etc/containers/systemd/users/
    \end{itemize}
  \end{exampleblock}
\end{frame}

\section{Unidades de systemd}

\begin{frame}[c]{Generador de Systemd}
  \begin{itemize}
    \item Podman permite la \textbf{creación e inicio} de contenedores
          (y la creación de volúmenes) a través de systemd
          mediante un \textbf{generador systemd}.
    \pausa
    \item Estos archivos se leen durante el arranque (y al ejecutar
      \textbf{systemctl daemon-reload}) y generan los archivos de
      unidad de \textbf{servicio systemd} correspondientes.
    \pausa
    \item El generador Podman lee las rutas de búsqueda anteriores y
      lee archivos con las extensiones
      \begin{itemize}
        \item .container
        \pausa
        \item .volume
        \pausa
        \item .network
        \pausa
        \item .build
        \pausa
        \item .pod
        \pausa
        \item .kube
      \end{itemize}
      y para cada archivo genera un archivo \textbf{.service} con el
      mismo nombre.
  \end{itemize}
\end{frame}

\begin{frame}[c]{Generador de Systemd}
  \begin{itemize}
    \item Los archivos de unidad generados se pueden iniciar y
          gestionar con \textbf{systemctl} como cualquier otro
          servicio systemd.
    \pausa
    \item \textbf{systemctl {--user} list-unit-files} enumera
          los archivos de unidades existentes en el sistema.
  \end{itemize}
\end{frame}

\begin{frame}[c]{Container units [Container]}

  Se utilizan para administrar contenedores mediante la ejecución del
  comando \textbf{podman run}.

  \vspace{\baselineskip}
  \begin{itemize}
    \item Extensión de archivo: \textbf{.container}
    \item Nombre de la sección: \textbf{[Container]}
    \item Campos requeridos: \textbf{Image} que describe la imagen del
      contenedor que ejecuta el servicio.
  \end{itemize}
\end{frame}

\begin{frame}[fragile]
  \frametitle{test.container}
  \lstinputlisting[basicstyle=\tiny]{ejemplo01/test.container}
\end{frame}

\begin{frame}[c]{Kube units [Kube]}
  Se utilizan para administrar contenedores definidos en archivos
  \textbf{YAML} de \textbf{Kubernetes} mediante la ejecución del
  comando \textbf{podman kube play}.
  \begin{itemize}
    \item Extensión de archivo: \textbf{.kube}
    \item Nombre de la sección: \textbf{[Kube]}
    \item Campos requeridos: \textbf{Yaml} que define la ruta al
      archivo YAML de \textbf{Kubernetes}.
  \end{itemize}
\end{frame}

\begin{frame}[fragile]
  \frametitle{test.kube}
  \lstinputlisting{ejemplo01/test.kube}
\end{frame}

\begin{frame}[c]{Network units [Network]}
  Se utilizan para crear \textbf{redes Podman} que pueden referenciarse en
  archivos .container o .kube.
  \begin{itemize}
    \item Extensión de archivo: \textbf{.network}
    \item Nombre de la sección: \textbf{[Network]}
    \item Campos requeridos: Ninguno
  \end{itemize}
\end{frame}

\begin{frame}[fragile]
  \frametitle{test.network}
  \lstinputlisting{ejemplo01/test.network}
\end{frame}

\begin{frame}[c]{Volume units [Volume]}
  Se utilizan para crear \textbf{volúmenes Podman} que pueden
  referenciarse en archivos .container.
  \begin{itemize}
    \item Extensión de archivo: \textbf{.volume}
    \item Nombre de la sección: \textbf{[Volume]}
    \item Campos requeridos: Ninguno
  \end{itemize}
\end{frame}

\begin{frame}[fragile]
  \frametitle{test.volume}
  \lstinputlisting{ejemplo01/test.volume}
\end{frame}

\begin{frame}[fragile]
  \frametitle{s3fs.volume}
  \lstinputlisting{ejemplo01/s3fs.volume}
  \href{https://github.com/s3fs-fuse/s3fs-fuse}
       {https://github.com/s3fs-fuse/s3fs-fuse}
\end{frame}

\begin{frame}[c]{Build units [Build]}
  Se utilizan para describir el comando para \textbf{construir}
  una imagen.
  \begin{itemize}
    \item Extensión de archivo: \textbf{.build}
    \item Nombre de la sección: \textbf{[Build]}
    \item Campos requeridos: \textbf{ImageTag} y cualquiera de
      \textbf{File} o \textbf{SetWorkingDirectory}
  \end{itemize}
\end{frame}

\begin{frame}[fragile]
  \frametitle{test.build}
  \lstinputlisting{ejemplo01/test.build}
  test.container
  \lstinputlisting{ejemplo01/test2.container}
\end{frame}

\begin{frame}[c]{Pod units [Pod]}
  Se utilizan para crear pods mediante la ejecución del
  comando \textbf{podman pod create}.
  \begin{itemize}
    \item Extensión de archivo: \textbf{.pod}
    \item Nombre de la sección: \textbf{[Pod]}
    \item Campos requeridos: Ninguno
  \end{itemize}
\end{frame}

\begin{frame}[fragile]
  \frametitle{test.pod}
  \lstinputlisting{ejemplo01/test.pod}
  centos.container
  \lstinputlisting{ejemplo01/centos.container}
\end{frame}

\section{Ejemplo: Proyecto Matefacil}

\begin{frame}[fragile]
  \frametitle{Volúmenes}

  Vemos si tenemos volúmenes creados

  \begin{lstlisting}
$ podman volume ls
$
  \end{lstlisting}

  No hay ningún volumen, vamos creando uno

  \pausa
  \begin{exampleblock}{Si es temporal:}
    \begin{lstlisting}
$ mkdir -p $XDG_RUNTIME_DIR/containers/systemd/
    \end{lstlisting}
  \end{exampleblock}
  \pausa
  \begin{block}{Si es permanente:}
    \begin{lstlisting}
$ mkdir -p $HOME/.config/containers/systemd/
    \end{lstlisting}
  \end{block}
\end{frame}

\begin{frame}[fragile]
  \frametitle{Volúmenes}
  Creo el archivo \textbf{matefacil-postgres.volume}
  \pausa
  \lstinputlisting{ejemplo02/matefacil-postgres.volume}
  \pausa
  Y después
  \begin{lstlisting}
$ systemctl --user daemon-reload
$
  \end{lstlisting}
  \pausa
  Pero ...
  \begin{lstlisting}
$ podman volume ls
$
  \end{lstlisting}
\end{frame}

\begin{frame}[fragile]
  \frametitle{Volúmenes}
  Ejecuto ...
  \begin{lstlisting}
$ systemctl --user list-unit-files
UNIT FILE                               STATE     PRESET 
...
matefacil-postgres-volume.service       generated -
...
$
  \end{lstlisting}
  \pausa
  Entonces ...
  \begin{lstlisting}
$ systemctl --user status matefacil-postgres-volume.service
o matefacil-postgres-volume.service
     Loaded: loaded (/run/user/1980/containers/systemd/matefacil-postgres.volume; generated)
     Active: inactive (dead)
$
  \end{lstlisting}
\end{frame}

\begin{frame}[fragile]
  \frametitle{Volúmenes}
  Ejecuto ...
  \begin{lstlisting}
$ systemctl --user start matefacil-postgres-volume.service
$
  \end{lstlisting}
  \pausa
  Y descubro ...
  \begin{lstlisting}
$ podman volume ls
DRIVER      VOLUME NAME
local       mfDBvol
$
  \end{lstlisting}
\end{frame}

\begin{frame}[fragile]
  \frametitle{Redes}
  Vemos las redes creadas
  \begin{lstlisting}
$ podman network ls
NETWORK ID    NAME        DRIVER
2f259bab93aa  podman      bridge
$
  \end{lstlisting}
  Creo el archivo \textbf{matefacil.network}
  \pausa
  \lstinputlisting{ejemplo02/matefacil.network}
\end{frame}

\begin{frame}[fragile]
  \frametitle{Redes}
  Y ejecuto ...
  \begin{lstlisting}
$ systemctl --user daemon-reload
$ systemctl --user list-unit-files
UNIT FILE                               STATE     PRESET
...
matefacil-network.service               generated -
matefacil-postgres-volume.service       generated -
...
$ systemctl --user status matefacil-network.service
o matefacil-network.service
     Loaded: loaded (/run/user/1980/containers/systemd/matefacil.network; generated)
     Active: inactive (dead)
$ systemctl --user start matefacil-network.service
  \end{lstlisting}
\end{frame}

\begin{frame}[fragile]
  \frametitle{Redes}
  Y finalmente ...
  \begin{lstlisting}
$ podman network ls
NETWORK ID    NAME        DRIVER
db2824b9d435  mfnet       bridge
2f259bab93aa  podman      bridge
$
  \end{lstlisting}
\end{frame}

\begin{frame}[fragile]
  \frametitle{Contenedor}
  Creo el archivo \textbf{matefacil-postgres.container}
  \lstinputlisting{ejemplo02/matefacil-postgres-mal.container}
\end{frame}

\begin{frame}[fragile]
  \frametitle{Contenedor}
  La parte de  \textbf{[Install]}, sirve para ejecutar el contenedor cuando
  inicia sesión el usuario.
  \lstinputlisting[firstline=1,lastline=2]{ejemplo02/matefacil-postgres-mal.container}
  \pausa
  \begin{alertblock}{Nota:}
    Si omitimos el \textbf{[Install]}, NO se iniciará el contenedor al
    iniciar sesión del usuario.
  \end{alertblock}
  Esto puede servir en nuestras maquinas de desarrollo.
\end{frame}

\begin{frame}[fragile]
  \frametitle{Contenedor}
  Usamos las otras unidades de systemd:
  \lstinputlisting[firstline=4,lastline=8]{ejemplo02/matefacil-postgres-mal.container}
\end{frame}

\begin{frame}[fragile]
  \frametitle{Contenedor}
  Ponemos las variables que usará Postgres:
  \lstinputlisting[linerange={4,9-11}]{ejemplo02/matefacil-postgres-mal.container}
  \pausa
  \begin{alertblock}{Pregunta:}
    ¿Alguien ve algo raro?
  \end{alertblock}
\end{frame}

\begin{frame}[fragile]
  \frametitle{Secretos}
  Podman puede manejar secretos 
  \begin{lstlisting}
podman secret COMANDO
  \end{lstlisting}

  \begin{itemize}
    \item \textbf{create} Crea un nuevo secreto
    \pausa
    \item \textbf{exists} Verifica si el secreto existe en el
      almacenamiento local
    \pausa
    \item \textbf{inspect} Inspecciona un secreto
    \pausa
    \item \textbf{ls} Lista los secretos
    \pausa
    \item \textbf{rm} Remueve uno o más secretos
  \end{itemize}
\end{frame}

\begin{frame}[fragile]
  \frametitle{Secretos}
  \lstinputlisting[basicstyle=\tiny]{ejemplo02/01-crear-secretos.sh}
\end{frame}

\begin{frame}[fragile]
  \frametitle{Secretos}
  Archivo \textbf{.env}
  \lstinputlisting{ejemplo02/ejemplo.env}
\end{frame}

\begin{frame}[fragile]
  \frametitle{Secretos}
  Verificamos que los secretos hayan sido guardados correctamente:
  \begin{lstlisting}[basicstyle=\tiny]
$ podman secret list
ID                         NAME               DRIVER      CREATED        UPDATED
00d96f478a401ef409ec85b7e  postgres-password  file        7 minutes ago  7 minutes ago
59579d85e5578ee0ebfc582c1  postgres-db        file        7 minutes ago  7 minutes ago
e9a2525df0b5bff47aac2beab  postgres-user      file        7 minutes ago  7 minutes ago
$
  \end{lstlisting}
\end{frame}

\begin{frame}[fragile]
  \frametitle{Contenedor}
  Usando secretos de Podman
  \lstinputlisting[linerange={4,9-11}]{ejemplo02/matefacil-postgres.container}
\end{frame}

\begin{frame}[fragile]
  \frametitle{Contenedor}
  Finalmente queda así:
  \lstinputlisting{ejemplo02/matefacil-postgres.container}
\end{frame}

\begin{frame}[fragile]
  \frametitle{Contenedor}
  Y ejecuto ...
  \begin{lstlisting}
$ systemctl --user daemon-reload
$ systemctl --user list-unit-files
UNIT FILE                               STATE     PRESET
...
matefacil-network.service               generated -
matefacil-postgres-volume.service       generated -
matefacil-postgres.service              generated -
...
$ systemctl --user start matefacil-postgres.service
$
  \end{lstlisting}
\end{frame}

\begin{frame}[fragile]
  \frametitle{Contenedor}
  Y verifico
  \begin{lstlisting}
$ podman ps
CONTAINER ID  IMAGE                            COMMAND 
c3cb478f05f0  docker.io/library/postgres:15.3  postgres
$
  \end{lstlisting}
  \begin{lstlisting}
CREATED        STATUS        PORTS       NAMES
6 minutes ago  Up 6 minutes  5432/tcp    matefacil-postgres
  \end{lstlisting}
\end{frame}

\begin{frame}[fragile]
  \frametitle{Contenedor}
  Y para detenerlo
  \begin{lstlisting}
$ systemctl --user stop matefacil-postgres.service
$ podman ps
CONTAINER ID  IMAGE  COMMAND   CREATED  STATUS  PORTS  NAMES
  \end{lstlisting}
\end{frame}


\begin{frame}{Agradecimientos y créditos}

  \begin{center}
    \huge ¡Gracias!
  \end{center}

  El código fuente de esta presentación la puedes encontrar en
  \begin{itemize}
    \item \href{https://github.com/miguelinux/slides}
      {https://github.com/miguelinux/slides}
      \{platicas/kcd25/\}

  \end{itemize}


  \vspace{\baselineskip}
  El material está basado en
  \begin{itemize}
    \item
      \href{https://docs.podman.io/en/latest/markdown/podman-systemd.unit.5.html}
      {podman-systemd.unit.5 (man podman-systemd.unit)}
    \item
      \href{https://docs.redhat.com/en/documentation/red\_hat\_enterprise\_linux/9/html/building\_running\_and\_managing\_containers/assembly\_porting-containers-to-systemd-using-podman\_building-running-and-managing-containers}
      {building\_running\_and\_managing\_containers (redhat)}
    \item
      \href{https://www.redhat.com/en/blog/container-systemd-persist-reboot}
            {container-systemd-persist-reboot (redhat)}
    \item
      \href{https://www.redhat.com/en/blog/multi-container-application-podman-quadlet}
      {multi-container-application-podman-quadlet (redhat)}
  \end{itemize}
\end{frame}

\end{document}
