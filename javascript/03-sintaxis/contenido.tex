% ex: ts=2 sw=2 sts=2 et filetype=tex
% SPDX-License-Identifier: CC-BY-SA-4.0

\section{Sintaxis}

\begin{frame}[fragile]
  \frametitle{Sintaxis de JavaScript}

  La sintaxis de JavaScript es el conjunto de reglas,
  cómo se construyen los programas de JavaScript:

  \vspace{\baselineskip}
  \begin{lstlisting}
// Como se crean las variables:
var x;
let y;

// Como se usan las variables:
x = 5;
y = 6;
let z = x + y;
  \end{lstlisting}
\end{frame}

\begin{frame}[c]{Valores de JavaScript}

  La sintaxis de JavaScript define dos tipos de valores:

  \begin{itemize}
    \item Valores \textbf{fijos}
    \item Valores \textbf{variables}
  \end{itemize}

  \vspace{\baselineskip}
  Los valores fijos se llaman \textbf{Literales}.

  \vspace{\baselineskip}
  Los valores de las variables se denominan \textbf{Variables}.
\end{frame}


\begin{frame}[fragile]
  \frametitle{Literales de JavaScript}

  Las dos reglas de sintaxis más importantes para valores fijos son:

  \begin{enumerate}
    \item Los \textbf{números} se escriben con o sin decimales:
          \vspace{\baselineskip}
          \begin{lstlisting}
10.50

1001
          \end{lstlisting}
    \item Las \textbf{cadenas} son texto, escrito entre comillas
          simples o dobles:
          \vspace{\baselineskip}
          \begin{lstlisting}
10.50

1001
          \end{lstlisting}
  \end{enumerate}
\end{frame}

\begin{frame}[fragile]
  \frametitle{Variables en JavaScript}

  En un lenguaje de programación, las variables se utilizan
  para \textbf{almacenar} valores de datos.

  \vspace{\baselineskip}
  JavaScript usa las palabras clave \textbf{var}, \textbf{let} y
  \textbf{const} para \textbf{declarar} variables.

  \vspace{\baselineskip}
  Se utiliza un \textbf{signo igual} para \textbf{asignar valores}
  a las variables.

  \vspace{\baselineskip}
  En este ejemplo, x se define como una variable.
  Entonces, a x se le asigna (se le da) el valor 6:

  \vspace{\baselineskip}
  \begin{lstlisting}
let x;

x = 6
  \end{lstlisting}
\end{frame}


\begin{frame}[fragile]
  \frametitle{Operadores en JavaScript}

  JavaScript usa \textbf{operadores aritméticos} ( + - * /) para
  \textbf{calcular} valores:

  \vspace{\baselineskip}
  \begin{lstlisting}
(5 + 6) * 10
  \end{lstlisting}

  \vspace{\baselineskip}
  JavaScript usa un \textbf{operador de asignación} ( = )
  para \textbf{asignar} valores a las variables:

  \vspace{\baselineskip}
  \begin{lstlisting}
let x, y;
x = 5;
y = 6;
  \end{lstlisting}
\end{frame}

\begin{frame}[fragile]
  \frametitle{Expresiones JavaScript}

  Una expresión es una combinación de valores,
  variables y operadores, que calcula un valor.

  \vspace{\baselineskip}
  El cálculo se llama \textbf{evaluación}.

  \vspace{\baselineskip}
  Por ejemplo, 5 * 10 se evalúa como 50:

  \vspace{\baselineskip}
  \begin{lstlisting}
5 * 10
  \end{lstlisting}

  \vspace{\baselineskip}
  Las expresiones también pueden contener valores de variables:

  \vspace{\baselineskip}
  \begin{lstlisting}
x * 10
  \end{lstlisting}
\end{frame}

\begin{frame}[fragile]
  \frametitle{Expresiones JavaScript}

  \vspace{\baselineskip}
  Los valores pueden ser de varios tipos, como números y cadenas.

  \vspace{\baselineskip}
  Por ejemplo, "Juan" + " " + "Pérez", se evalúa como "Juan Pérez":

  \vspace{\baselineskip}
  \begin{lstlisting}
"Juan" + " " + "Pérez"
  \end{lstlisting}
\end{frame}

\begin{frame}[fragile]
  \frametitle{Palabras clave de JavaScript}

  Las palabras clave de JavaScript se utilizan para
  identificar las acciones a realizar.

  \vspace{\baselineskip}
  La palabra clave \textbf{let} le dice al navegador que cree variables:

  \vspace{\baselineskip}
  \begin{lstlisting}
let x, y;
x = 5 + 6;
y = x * 10;
  \end{lstlisting}
\end{frame}

\begin{frame}[fragile]
  \frametitle{Palabras clave de JavaScript}

  La palabra clave \textbf{var} también le dice
  al navegador que cree variables:

  \vspace{\baselineskip}
  \begin{lstlisting}
var x, y;
x = 5 + 6;
y = x * 10;
  \end{lstlisting}

  \begin{block}{}
  En estos ejemplos, usar var o let producirá el mismo resultado.
  Aprenderás más sobre var y let más adelante en este curso.
  \end{block}
\end{frame}

\begin{frame}[fragile]
  \frametitle{Comentarios en JavaScript}

  No todas las declaraciones de JavaScript se "ejecutan".

  \vspace{\baselineskip}
  El código después de barras oblicuas dobles // o entre
  /* y */se trata como un comentario .

  \vspace{\baselineskip}
  Los comentarios se ignoran y no se ejecutarán:

  \vspace{\baselineskip}
  \begin{lstlisting}
let x = 5;   // Este si se ejecuta

// x = 6;   Este no se ejecutará
  \end{lstlisting}

  \begin{block}{}
  Aprenderemos más sobre los comentarios en un capítulo posterior.
  \end{block}
\end{frame}

\begin{frame}[c]{Identificadores/Nombres de JavaScript}

  Los identificadores son nombres de JavaScript.

  \vspace{\baselineskip}
  Los identificadores se utilizan para nombrar variables y
  palabras clave y funciones.

  \vspace{\baselineskip}
  Las reglas para los nombres legales son las mismas en la
  mayoría de los lenguajes de programación.

  \vspace{\baselineskip}
  Un nombre de JavaScript debe comenzar con:

  \begin{itemize}
    \item Una letra (A-Z o a-z)
    \item Un signo de peso/dólar (\$)
    \item Un guión bajo (\_)
  \end{itemize}

  \vspace{\baselineskip}
  Los caracteres subsiguientes pueden ser letras, dígitos,
  guiones bajos o signos de peso/dólar.
\end{frame}

\begin{frame}[c]{Identificadores/Nombres de JavaScript}
  \begin{alertblock}{Nota:}
    No se permiten números como primer carácter en los nombres.

    \vspace{\baselineskip}
    De esta forma, JavaScript puede distinguir fácilmente los identificadores de los números.
  \end{alertblock}
\end{frame}

\begin{frame}[fragile]
  \frametitle{JavaScript distingue entre mayúsculas y minúsculas}

  Todos los identificadores de JavaScript \textbf{distinguen
  entre mayúsculas y minúsculas}.

  \vspace{\baselineskip}
  Las variables nombre y Nombre, son dos variables diferentes:

  \vspace{\baselineskip}
  \begin{lstlisting}
let nombre, Nombre;
nombre = "Maria";
Nombre = "Juan";
  \end{lstlisting}

  \vspace{\baselineskip}
  JavaScript no interpreta LET o Let como la palabra clave let.
\end{frame}

\begin{frame}[c]{Variables con múltiples palabras}
  Un nombre de una variable con múltiples palabras puede ser difícil de leer.
  Hay varias técnicas que se pueden usar para que sean mas legibles.
  \pausa
  \begin{block}{Pascal case}
    Cada palabra comienza con una mayúscula:
    \textbf{ElNombreDeMiVariable}
  \end{block}
  \pausa
  \begin{block}{Snake case}
    Cada palabra esta separada por un guión bajo:
    \textbf{el\_nombre\_de\_mi\_variable}
  \end{block}
  \pausa
  \begin{exampleblock}{Camel case}
    Cada palabra, excepto la primera, comienza con una mayúscula:
    \textbf{elNombreDeMiVariable}

    \vspace{\baselineskip}
    Los programadores de JavaScript usan la técnica \textbf{Camel Case}.
  \end{exampleblock}
\end{frame}

\begin{frame}[c]{Conjunto de caracteres JavaScript}

  JavaScript utiliza el conjunto de caracteres \textbf{Unicode}.

  \vspace{\baselineskip}
  Unicode cubre (casi) todos los caracteres, puntuaciones y símbolos del mundo.
\end{frame}

\section{Comentarios}

\begin{frame}[c]{Comentarios en JavaScript}

  Los comentarios de JavaScript se pueden usar para explicar
  el código de JavaScript y hacerlo más legible.

  \vspace{\baselineskip}
  Los comentarios de JavaScript también se pueden usar para
  evitar la ejecución, al probar código alternativo.
\end{frame}

\begin{frame}[fragile]
  \frametitle{Comentarios de una sola línea}

  Los comentarios de una sola línea comienzan con //.

  \vspace{\baselineskip}
  Cualquier texto entre // y el final de la línea será ignorado
  por JavaScript (no se ejecutará).

  \vspace{\baselineskip}
  Este ejemplo usa un comentario de una sola línea antes de
  cada línea de código:

  \vspace{\baselineskip}
  \begin{lstlisting}
// Cambiando el encabezado:
document.getElementById("miEncabezado").innerHTML = "Mi primera página";

// Cambiando el parrafo:
document.getElementById("miParrafo").innerHTML = "Mi primer párrafo.";
  \end{lstlisting}
\end{frame}

\begin{frame}[fragile]
  \frametitle{Comentarios de una sola línea}

  Este ejemplo utiliza un comentario de una sola línea al
  final de cada línea para explicar el código:

  \vspace{\baselineskip}
  \begin{lstlisting}
let x = 5;      // Declara x y le asigna el valor de 5
let y = x + 2;  // Declara y y le asigna el valor de x + 2
  \end{lstlisting}
\end{frame}

\begin{frame}[fragile]
  \frametitle{Comentarios de varias líneas}

  Los comentarios de varias líneas comienzan con /* y
  terminan con */.

  \vspace{\baselineskip}
  Cualquier texto entre /* y */será ignorado por JavaScript.

  \vspace{\baselineskip}
  Este ejemplo utiliza un comentario de varias líneas
  (un bloque de comentarios) para explicar el código:

  \vspace{\baselineskip}
  \begin{lstlisting}
/*
   El siguiente código cambia el texto de un
   encabezado (H1, H2, etc.) con id = "miEncabezado" y un
   parrafo (p) con id = "miParrafo" en la pagina web
*/
document.getElementById("miEncabezado").innerHTML = "Mi primera página";
document.getElementById("miParrafo").innerHTML = "Mi primer párrafo.";
  \end{lstlisting}
\end{frame}

\begin{frame}[c]{Comentarios de varias líneas}
  \begin{block}{Nota:}
    Lo más común es utilizar comentarios de una sola línea.
    Los comentarios en bloque se utilizan a menudo para la
    documentación formal.
  \end{block}
\end{frame}

\begin{frame}[fragile]
  \frametitle{Uso de comentarios para evitar la ejecución}

  El uso de comentarios para evitar la ejecución de código es
  adecuado para las pruebas de código.

  \vspace{\baselineskip}
  Agregar // delante de una línea de código cambia las líneas de
  código de una línea ejecutable a un comentario.

  \vspace{\baselineskip}
  Este ejemplo utiliza // para evitar la ejecución de una de las
  líneas de código:

  \vspace{\baselineskip}
  \begin{lstlisting}
//document.getElementById("miEncabezado").innerHTML = "Mi primera página";
document.getElementById("miParrafo").innerHTML = "Mi primer párrafo.";
  \end{lstlisting}
\end{frame}

\begin{frame}[fragile]
  \frametitle{Uso de comentarios para evitar la ejecución}

  Este ejemplo utiliza un bloque de comentarios para evitar
  la ejecución de varias líneas:

  \vspace{\baselineskip}
  \begin{lstlisting}
/*
document.getElementById("miEncabezado").innerHTML = "Mi primera página";
document.getElementById("miParrafo").innerHTML = "Mi primer párrafo.";
*/
  \end{lstlisting}
\end{frame}
