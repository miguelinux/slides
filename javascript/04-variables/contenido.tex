% ex: ts=2 sw=2 sts=2 et filetype=tex
% SPDX-License-Identifier: CC-BY-SA-4.0

\section{Variables en JavaScript}

\begin{frame}[c]{Variables en JavaScript}

  Hay 4 formas de declarar una variable de JavaScript:

  \vspace{\baselineskip}
  \begin{itemize}
    \item Utilizando \eti{var}
    \item Utilizando \eti{let}
    \item Utilizando \eti{const}
    \item Usando nada
  \end{itemize}

  \vspace{\baselineskip}
  En seguida explicaremos cada una de ellas.
\end{frame}

\begin{frame}[fragile]
  \frametitle{¿Qué son las Variables?}

  Las variables son contenedores para almacenar
  datos (almacenar valores de datos).

  \vspace{\baselineskip}
  En este ejemplo \eti{x}, \eti{y} y \eti{z}, son variables
  declaradas con la palabra clave \eti{var}:

  \vspace{\baselineskip}
  \begin{lstlisting}
var x = 5;
var y = 6;
var z = x + y;
  \end{lstlisting}
\end{frame}

\begin{frame}[fragile]
  \frametitle{¿Qué son las Variables?}

  En este ejemplo \eti{x}, \eti{y} y \eti{z}, son variables
  declaradas con la palabra clave \eti{let}:

  \vspace{\baselineskip}
  \begin{lstlisting}
let x = 5;
let y = 6;
let z = x + y;
  \end{lstlisting}
\end{frame}

\begin{frame}[fragile]
  \frametitle{¿Qué son las Variables?}

  En este ejemplo \eti{x}, \eti{y} y \eti{z}, son variables
  no declaradas:

  \vspace{\baselineskip}
  \begin{lstlisting}
x = 5;
y = 6;
z = x + y;
  \end{lstlisting}
\end{frame}

\begin{frame}[c]{¿Qué son las Variables?}

  De todos los ejemplos anteriores, se puede observar que:

  \vspace{\baselineskip}
  \begin{itemize}
    \item x almacena el valor 5
    \item y almacena el valor 6
    \item z almacena el valor 11
  \end{itemize}
\end{frame}

\begin{frame}[c]{¿Cuándo usar \eti{var} en JavaScript?}

  Declare siempre las variables de JavaScript con \eti{var},
  \eti{let} o \eti{const}.

  \vspace{\baselineskip}
  La palabra clave \eti{var} se usa en todo el código JavaScript
  desde 1995 hasta 2015.

  \vspace{\baselineskip}
  Las palabras clave \eti{let} y \eti{const} se agregaron
  a JavaScript en 2015.

  \vspace{\baselineskip}
  Si deseas que tu código se ejecute en un navegador anterior,
  debes usar \eti{var}.
\end{frame}

\begin{frame}[fragile]
  \frametitle{¿Cuándo usar \eti{const} en JavaScript?}

  Si quieres una regla general: declara siempre las variables
  con \eti{const}.

  \vspace{\baselineskip}
  Si cree que el valor de la variable puede cambiar, use \eti{let}.

  \vspace{\baselineskip}
  En este ejemplo \atri{price1}, \atri{price2} y \atri{total}, son variables:

  \vspace{\baselineskip}
  \begin{lstlisting}
const price1 = 5;
const price2 = 6;
let total = price1 + price2;
  \end{lstlisting}
\end{frame}

\begin{frame}[c]{¿Cuándo usar \eti{const} en JavaScript?}

  Las dos variables \atri{price1} y \atri{price2} se
  declaran con la palabra clave \eti{const}.

  \vspace{\baselineskip}
  Estos son valores constantes y no se pueden cambiar.

  \vspace{\baselineskip}
  La variable \atri{total} se declara con la palabra clave \eti{let}.

  \vspace{\baselineskip}
  Este es un valor que se puede cambiar.
\end{frame}

\begin{frame}[fragile]
  \frametitle{Al igual que el álgebra}

  Al igual que en álgebra, las variables tienen valores:

  \vspace{\baselineskip}
  \begin{lstlisting}
let x = 5;
let y = 6;
  \end{lstlisting}

  \vspace{\baselineskip}
  Al igual que en álgebra, las variables se usan en expresiones:

  \vspace{\baselineskip}
  \begin{lstlisting}
let z = x + y;
  \end{lstlisting}

  \vspace{\baselineskip}
  Del ejemplo anterior, puede ver que el total se calcula en 11.

  \begin{exampleblock}{Nota:}
    Las variables son contenedores para almacenar valores.
  \end{exampleblock}
\end{frame}

\begin{frame}[c]{Identificadores JavaScript}

  \textbf{Todas las variables} de JavaScript deben
  \textbf{identificarse} con \textbf{nombres únicos}.

  \vspace{\baselineskip}
  Estos nombres únicos se denominan \textbf{identificadores}.

  \vspace{\baselineskip}
  Los identificadores pueden ser nombres cortos (como x e y)
  o nombres más descriptivos (edad, suma, volumen total).
\end{frame}

\begin{frame}[c]{Identificadores de JavaScript}

  Las reglas generales para construir nombres para variables
  (identificadores únicos) son:

  \vspace{\baselineskip}
  \begin{itemize}
    \item Los nombres pueden contener letras, dígitos,
          guiones bajos y signos de peso/dólar.
    \item Los nombres deben comenzar con una letra.
    \item Los nombres también pueden comenzar con \$ y \_
          (pero no lo usaremos en este curso)
    \item Los nombres distinguen entre mayúsculas y
          minúsculas (y e Y son variables diferentes)
    \item Las palabras reservadas (como las palabras clave de
          JavaScript) no se pueden usar como nombres de variables
  \end{itemize}

  \vspace{\baselineskip}
  \begin{block}{Nota:}
    Los identificadores de JavaScript distinguen entre
    mayúsculas y minúsculas.
  \end{block}
\end{frame}

\begin{frame}[fragile]
  \frametitle{El operador de asignación}

  En JavaScript, el signo igual (\atri{=}) es un operador de
  "asignación", no un operador de comparación como en "igual a".

  \vspace{\baselineskip}
  Esto es diferente del álgebra. Lo siguiente no tiene sentido en álgebra:

  \vspace{\baselineskip}
  \begin{lstlisting}
x = x + 5
  \end{lstlisting}

  En JavaScript, sin embargo, tiene mucho sentido: asigna el valor de x + 5 a x.
  (Calcula el valor de x + 5 y pone el resultado en x. El valor de x se incrementa en 5).

  \begin{exampleblock}{Nota:}
    El operador de comparación "igual a" se escribe como
    \atri{==} en JavaScript.
  \end{exampleblock}
\end{frame}

\begin{frame}[fragile]
  \frametitle{Tipos de datos de JavaScript}

  Las variables de JavaScript pueden contener números como
  100 y valores de texto como "Juan Perez".

  \vspace{\baselineskip}
  En programación, los valores de texto se denominan \textbf{cadenas de
  texto}.

  \vspace{\baselineskip}
  JavaScript puede manejar muchos tipos de datos, pero por ahora,
  solo piense en números y cadenas.

  \vspace{\baselineskip}
  Las cadenas se escriben entre comillas simples o dobles.
  Los números se escriben sin comillas.

  \vspace{\baselineskip}
  Si pone un número entre comillas, se tratará como una cadena de texto.

  \vspace{\baselineskip}
  \begin{lstlisting}
const pi = 3.1416;
let person = "Juan Perez";
let answer = '¡Si, yo soy!';
  \end{lstlisting}
\end{frame}

\begin{frame}[fragile]
  \frametitle{Declarar una variable de JavaScript}

  Crear una variable en JavaScript se llama "\textbf{declarar}"
  una variable.

  \vspace{\baselineskip}
  Declaras una variable de JavaScript con la palabra clave \eti{var}
  o \eti{let}:

  \vspace{\baselineskip}
  \begin{lstlisting}
var nombre;
  \end{lstlisting}
o
  \begin{lstlisting}
let nombre;
  \end{lstlisting}

  \vspace{\baselineskip}
  Después de la declaración, la variable no tiene valor
  (técnicamente se define como \atri{undefined}).
\end{frame}

\begin{frame}[fragile]
  \frametitle{Declarar una variable de JavaScript}
  Para asignar un valor a la variable, utilice el signo igual:
  \vspace{\baselineskip}
  \begin{lstlisting}
nombre = "Maria";
  \end{lstlisting}

  \vspace{\baselineskip}
  También puede asignar un valor a la variable cuando la declara:

  \vspace{\baselineskip}
  \begin{lstlisting}
let nombre = "Maria";
  \end{lstlisting}
\end{frame}

\begin{frame}[fragile]
  \frametitle{Declarar una variable de JavaScript}

  En el siguiente ejemplo, creamos una variable llamada
  \atri{nombre} y le asignamos el valor "Maria".

  \vspace{\baselineskip}
  Luego, "damos salida" al valor dentro de un párrafo HTML con id="demo":

  \vspace{\baselineskip}
  \begin{lstlisting}
<p id="demo"></p>

<script>
let nombre = "Maria";
document.getElementById("demo").innerHTML = nombre;
</script> 
  \end{lstlisting}

  \begin{exampleblock}{Nota:}
    Es una buena práctica de programación declarar todas
    las variables al comienzo de un script.
  \end{exampleblock}
\end{frame}

\begin{frame}[fragile]
  \frametitle{Una declaración, muchas variables}

  Puede declarar muchas variables en una declaración.

  \vspace{\baselineskip}
  Comience la declaración con \eti{let} y separe las variables
  con \textbf{comas}:

  \vspace{\baselineskip}
  \begin{lstlisting}
let persona = "Juan Perez", carro = "Volvo", precio = 200; 
  \end{lstlisting}

  \vspace{\baselineskip}
  Una declaración puede abarcar varias líneas:

  \vspace{\baselineskip}
  \begin{lstlisting}
let persona = "Juan Perez",
    carro = "Volvo",
    precio = 200; 
  \end{lstlisting}
\end{frame}

\begin{frame}[fragile]
  \frametitle{Valor = indefinido}

  En los programas de computadora, las variables a menudo se declaran
  sin valor. El valor puede ser algo que debe calcularse o algo que
  se proporcionará más adelante, como una entrada del usuario.

  \vspace{\baselineskip}
  Una variable declarada sin valor tendrá el valor \atri{undefined}.

  \vspace{\baselineskip}
  La variable \textbf{nombre} tendrá el valor \atri{undefined} después de la
  ejecución de esta sentencia:

  \vspace{\baselineskip}
  \begin{lstlisting}
let nombre;
  \end{lstlisting}
\end{frame}

\begin{frame}[fragile]
  \frametitle{Redeclaración de variables de JavaScript}

  Si vuelve a declarar una variable de JavaScript declarada
  con \eti{var}, no perderá su valor.

  \vspace{\baselineskip}
  La variable \textbf{carro} seguirá teniendo el valor
  "Volvo" después de la ejecución de estas sentencias:

  \vspace{\baselineskip}
  \begin{lstlisting}
var carro = "Volvo";
var carro;
  \end{lstlisting}
\end{frame}

\begin{frame}[fragile]
  \frametitle{Redeclaración de variables de JavaScript}

  \begin{alertblock}{Nota:}
    No puede volver a declarar una variable declarada con
    \eti{let} o \eti{const}.

    \vspace{\baselineskip}
    Esto no funcionará:
    \begin{lstlisting}
let carro = "Volvo";
let carro;
    \end{lstlisting}
  \end{alertblock}
\end{frame}

\begin{frame}[fragile]
  \frametitle{Aritmética en JavaScript}

  Al igual que con el álgebra, puedes hacer operaciones
  aritméticas con variables de JavaScript,
  usando operadores como \atri{=} y \atri{+}:

  \vspace{\baselineskip}
  \begin{lstlisting}
let x = 5 + 2 + 3;
  \end{lstlisting}

  \vspace{\baselineskip}
  También puede agregar cadenas, pero las cadenas se concatenarán:
  \begin{lstlisting}
let x = "Juan" + "  " + "Perez";
  \end{lstlisting}
\end{frame}

\begin{frame}[fragile]
  \frametitle{Aritmética en JavaScript}

  Prueba también esto:
  \vspace{\baselineskip}
  \begin{lstlisting}
let x = "5" + 2 + 3;
  \end{lstlisting}

  \vspace{\baselineskip}
  \begin{alertblock}{Nota:}
    Si pone un número entre comillas, el resto de los números se
    tratarán como cadenas y se concatenarán.
  \end{alertblock}

  \vspace{\baselineskip}
  Ahora prueba esto:
  \begin{lstlisting}
let x = 2 + 3 + "5";
  \end{lstlisting}
\end{frame}

\begin{frame}[fragile]
  \frametitle{Aritmética en JavaScript}
  \lstinputlisting{04-ejemplo01.html}
\end{frame}

\begin{frame}[fragile]
  \frametitle{Signo de pesos/dólar en JavaScript \$}

  Dado que JavaScript trata un signo de pesos/dólar como
  una letra, los identificadores que contienen \textbf{\$} son
  nombres de variables válidos:

  \vspace{\baselineskip}
  \begin{lstlisting}
let $ = "Hola Mundo";
let $$ = 2;
let $miDinero = 5;
  \end{lstlisting}

  \vspace{\baselineskip}
  Usar el signo de pesos/dólar no es muy común en JavaScript,
  pero los programadores profesionales a menudo lo usan como
  un alias para la función principal en una biblioteca de
  JavaScript.

  \vspace{\baselineskip}
  En la biblioteca JavaScript jQuery, por ejemplo, la función
  principal se llama \eti{\$} se usa para seleccionar elementos
  HTML. En jQuery \eti{\$("p");} significa "seleccionar todos
  los elementos p".
\end{frame}

\begin{frame}[fragile]
  \frametitle{Guión bajo en JavaScript (\_)}

  Dado que JavaScript trata el guión bajo como una letra,
  los identificadores que contienen \textbf{\_} son nombres
  de variables válidos:

  \vspace{\baselineskip}
  \begin{lstlisting}
let _nombre = "Elena";
let _x = 2;
let _100 = 5;
let _ = 3.5;
  \end{lstlisting}

  \vspace{\baselineskip}
  Usar el guión bajo no es muy común en JavaScript, pero una
  convención entre los programadores profesionales es usarlo
  como un alias para las variables "privadas (ocultas)".
\end{frame}
