% ex: ts=2 sw=2 sts=2 et filetype=tex
% SPDX-License-Identifier: CC-BY-SA-4.0

\section{Autenticación de múltiples factores (MFA)}

\begin{frame}[c]{La autenticación de múltiples factores (AMF)}
  \begin{itemize}
    \item Más comúnmente conocida por sus siglas en inglés \textbf{MFA}
      (\emph{Multi Factor Authentication})

    \pausa
    \item Es un método de \textbf{control de acceso informático} en el
      que a un usuario se le concede acceso al sistema solo después de que
      presente \textbf{dos o más pruebas diferentes} de que es quien dice ser.

    \pausa
    \item Estas pruebas pueden ser diversas, como una contraseña, que
      posea una \textbf{clave secundaria rotativa}, o un \textbf{certificado
      digital} instalado en el equipo, \emph{biometría}, entre otros.
  \end{itemize}
\end{frame}

\begin{frame}[c]{La autenticación de dos factores (A2F)}
  \begin{itemize}
    \item También usada la sigla inglesa \textbf{2FA} (\emph{de two-factor
      authentication})

    \pausa
    \item Es un \underline{método} que confirma que un \underline{usuario
      es quien dice ser} combinando dos componentes diferentes de entre:
      \begin{enumerate}
        \item algo que saben
        \item algo que tienen
        \item algo que son
      \end{enumerate}

    \pausa
    \item Es el \textbf{método más extendido en la actualidad} para acceder a
      cuentas de correo y otra aplicaciones, pero generalmente se solicita que
      el \textbf{usuario active voluntariamente esta capa de protección
      adicional}.
  \end{itemize}
\end{frame}

\begin{frame}[c]{Ejemplo de 2FA}
    \begin{columns}
        \column{0.5\textwidth}
        \begin{itemize}
          \item Un ejemplo de la vida cotidiana de este tipo de
            autenticación es la retirada de efectivo de un \textbf{cajero
            automático}.
          \pausa
          \item Solo tras combinar una tarjeta de crédito —\textbf{algo que
            el usuario posee}— y un pin —\textbf{algo que el usuario sabe}—
            se permite que la transacción se lleve a cabo.
        \end{itemize}
        \column{0.5\textwidth}
        \begin{center}
            \includegraphics[scale=0.5]{lfs/2fa-cajero.png}
        \end{center}
    \end{columns}
\end{frame}

\begin{frame}[c]{La autenticación de dos factores (A2F)}
  \begin{block}{}
      La \textbf{autenticación en dos pasos} o \textbf{verificación en
      dos pasos} es un método de confirmar la identidad de un usuario
      utilizando algo que \textbf{conocen} (contraseña) y un segundo
      factor distinto a lo que \textbf{sean} o \textbf{posean}.
  \end{block}

  \pausa
  \vspace{\baselineskip}
  Un ejemplo de un segundo paso es que el usuario tenga que introducir algo
  que le sea \textbf{enviado a través de un medio alternativo},
  o que tenga que introducir una serie de \textbf{dígitos generados por una
  aplicación} conocida por el usuario y el sistema de autenticación. 
\end{frame}

\begin{frame}[c]{Factores de autenticación}

  Los factores de autenticación de un patrón de autenticación de
  múltiples factores podría incluir:

  \vspace{\baselineskip}
  \begin{itemize}
    \item Algún objeto físico en \textbf{posesión del usuario}, como una
      memoria USB con un identificador único, una tarjeta de crédito,
      una llave, etc.

    \pausa
    \item Algún \textbf{secreto conocido} por el usuario, como una contraseña,
      un pin, etc.

    \pausa
    \item Alguna \textbf{característica biométrica propia del usuario},
      como una huella dactilar, iris, voz, velocidad de escritura,
      patrón en los intervalos de pulsación de teclas, etc.
  \end{itemize}
\end{frame}

\begin{frame}[c]{2FA vs. MFA: ¿Cuál es la diferencia?}
  \begin{itemize}
    \item La autenticación de dos factores (2FA) requiere que los
      usuarios presenten \textbf{dos tipos de autenticación}

    \pausa
    \item La autenticación de factores múltiples (MFA) requiere que
      los usuarios presenten al menos \textbf{dos, si no más},
      tipos de autenticación.
  \end{itemize}
\end{frame}

\begin{frame}[c]{Qué métodos de verificación en dos pasos hay}
  \begin{description}
    \item [Verificación por SMS] Tras escribir tu nombre y contraseña,
      se te envía una contraseña temporal por SMS para que la escribas.
    \pausa
    \item [Verificación por correo electrónico] Tras escribir tu nombre y
      contraseña, se te envía una contraseña temporal por correo electrónico. 
    \pausa
    \item [Pregunta de seguridad] Tras escribir tu nombre y contraseña,
      tendrás que responder a una pregunta de seguridad cuyas respuestas
      hayas configurado antes. 
    \pausa
    \item [Aplicaciones de autenticación] Tras escribir tu nombre y contraseña,
      tendrás que verificar tu identidad con la contraseña temporal que se te
      genera en la aplicación de verificación en dos pasos que tengas
      configurada. 
  \end{description}
\end{frame}

\begin{frame}[c]{Qué métodos de verificación en dos pasos hay}
  \begin{description}
    \item [Códigos en la propia app] Algunas aplicaciones móviles de servicios
      como Facebook también tienen función de generar códigos para cuando
      inicies sesión en nuevos dispositivos. 
    \pausa
    \item [Llaves de seguridad] Puedes crear una llave de seguridad USB,
      aunque también puede ser NFC o Bluetooth.
    \pausa
    \item [Biometría] Muy común en móviles, tabletas y que se ve en
      ordenadores y portátiles a veces. 
    \pausa
    \item [Códigos de recuperación] Cuando te identificas en algunos servicios,
      estos pueden generar un código de recuperación que te piden que anotes
      o imprimas. 
  \end{description}
\end{frame}

\section{Algoritmos para la autenticación a 2 pasos}

\begin{frame}[c]{Iniciativa para la Autenticación Abierta}
  \textbf{Initiative for Open Authentication} (\textbf{OATH}) es una
  colaboración a nivel de toda la industria para desarrollar una arquitectura
  de referencia abierta que utilice estándares abiertos para promover la
  adopción de una \textbf{autenticación fuerte}. 

  \vspace{\baselineskip}
  La OATH no está relacionada con la \textbf{OAuth}, un estándar
  abierto de autorización. 
\end{frame}

\begin{frame}[c]{Autenticación con contraseña de un solo uso}

  Una contraseña de un solo uso u \textbf{OTP} (del inglés
  \textbf{One-Time Password}) es una contraseña que pierde su
  validez después de su uso, de ahí su denominación.

  \vspace{\baselineskip}
  Por lo general, se emplea como parte de una autenticación de
  doble factor.
\end{frame}

\begin{frame}[c]{HMAC}
  En la criptografía, un \textbf{HMAC} (a veces expandido como
  \textbf{código de autentificación de mensajes en clave-hash} o
  \textbf{código de autenticación de mensaje basado en hash}) es una
  construcción específica para calcular un \underline{código de
  autentificación de mensaje (MAC)} que implica una función
  \textbf{hash criptográfica} en combinación con una \textbf{llave
  criptográfica secreta}.

  \vspace{\baselineskip}
  Como cualquier MAC, puede ser utilizado para verificar simultáneamente
  la integridad de los datos y la autentificación de un mensaje. 
\end{frame}

\begin{frame}[c]{Algoritmo de contraseña de un solo uso basado en HMAC}
  \textbf{HOTP} es un algoritmo de autenticación basado en \textbf{HMAC}
  y \textbf{one-time password}, que nació por la necesidad de crear un
  mecanismo de autenticación robusto y que supliese a los esquemas de
  autenticación débiles.

  \vspace{\baselineskip}
  Con HOTP se necesita de un segundo factor de autenticación, que solo
  es válido para una sola vez, OTP (One Time Password).

  \vspace{\baselineskip}
  La IETF publicó HOTP en el \textbf{RFC 4226} en diciembre de 2005.
\end{frame}

\begin{frame}[c]{Algoritmo de contraseña de un solo uso basado en HMAC}
  El algoritmo \textbf{HOTP} se basa en un valor de un \textbf{contador
  creciente} y una \underline{clave simétrica} conocida solo por el
  usuario y el servidor.

  \vspace{\baselineskip}
  Con el fin de crear el valor HOTP, se usa HMAC, como puede ser HMAC-SHA1.
  Usando este algoritmo de hash, lo que se hace es generar un token usando
  el algoritmo HMAC elegido.

  \begin{description}
    \item[K] Clave simétrica
    \item[C] Contador, que cuenta las iteraciones
    \item[HMAC-SHA1(K,C)] Genera el token
  \end{description}

  \vspace{\baselineskip}
  Una vez tenemos el token, lo enviamos al servidor, que hará la misma
  operación, y si el token es válido autentica al usuario en el sistema.
\end{frame}

\begin{frame}[c]{Algoritmo de contraseña de un solo uso basada en el tiempo}
  El algoritmo de contraseña de un solo uso o \textbf{TOTP} (Time-based one
  time password en inglés) es un algoritmo que permite generar una contraseña
  de un solo uso (\textbf{OTP}) que utiliza la \textbf{hora actual}
  como fuente de singularidad.

  \vspace{\baselineskip}
  Es una extensión del algoritmo de contraseña de un solo uso basado en
  \textbf{HMAC} (HOTP).

  \vspace{\baselineskip}
  El algoritmo TOTP es la base de la Iniciativa para la
  autenticación abierta (OATH).
  En mayo de 2011, el algoritmo TOTP se convirtió oficialmente en
  el estándar \textbf{RFC 6238}
\end{frame}

\begin{frame}[c]{Algoritmo TOTP}
  Para establecer la autenticación \textbf{TOTP}, el usuario y el
  autenticador deben disponer tanto de los \textbf{parámetros HOTP}
  como los siguientes parámetros propios:

  \begin{description}
    \item [$T_0$] el tiempo Unix a partir de la cual empezar a contar
      las unidades de tiempo (por defecto es $0$).
    \item [$T_X$] un intervalo que se utilizará para calcular el valor
      del contador $C_T$ (por defecto es 30 segundos).
  \end{description}
\end{frame}

\begin{frame}[c]{Algoritmo TOTP}
  TOTP usa internamente el algoritmo \textbf{HOTP}, reemplazando
  el contador con un valor no decreciente basado en la hora actual:
  \begin{displaymath}
    V_{TOTP}(K) = V_{HOTP}(K, C_T)
  \end{displaymath}
  Donde el valor del contador se calcula como:
  \begin{displaymath}
    C_T = \left\lfloor \frac{T - T_0}{T_X} \right\rfloor
  \end{displaymath}
  \begin{description}
    \item [$C_T$] es el recuento del número de intervalos $T_X$ entre
      $T_0$ y $T$.
    \item [$T$] es el tiempo actual en segundos desde una época en particular.
    \item [$T_0$] es la época especificada en segundos contando a partir
      de la época Unix (si se usa \textbf{Tiempo Unix}, entonces $T_0$ será
      $0$).
    \item [$T_X$] es la duración de un intervalo de tiempo (por ejemplo,
      \textbf{30 segundos}).
  \end{description}
\end{frame}

\section{Aplicaciones para la autenticación a 2 pasos}

\begin{frame}[c]{LinOTP}

  \textbf{LinOTP}: la solución de código abierto para la
  autenticación multifactor

  \vspace{\baselineskip}
  LinOTP es verdaderamente abierto en dos sentidos.
  Sus módulos y componentes tienen licencia AGPLv3 y le
  brindan una solución de código abierto completa y
  funcional para una autenticación multifactor sólida.

  \vspace{\baselineskip}
  \href{https://github.com/LinOTP/LinOTP}{https://github.com/LinOTP/LinOTP}
\end{frame}

\begin{frame}[c]{Aegis Authenticator}
  \textbf{Aegis Authenticator} es una aplicación 2FA gratuita, segura y
  de código abierto para Android. Su objetivo es proporcionar un
  autenticador seguro para sus servicios en línea, al tiempo que incluye 
  lgunas funciones que faltan en las aplicaciones de autenticación
  existentes, como cifrado y copias de seguridad adecuados.

  \vspace{\baselineskip}
  Aegis admite HOTP y TOTP, lo que lo hace compatible con miles de servicios.

  \vspace{\baselineskip}
  \href{https://github.com/beemdevelopment/Aegis}{https://github.com/beemdevelopment/Aegis}
\end{frame}

\begin{frame}[c]{FreeOTP}

  \textbf{FreeOTP} es un de software gratuito y de código abierto generador
  de tokens que se puede utilizar para la autenticación de dos factores.

  \vspace{\baselineskip}
  Proporciona implementaciones de HOTP y TOTP. Los tokens se pueden agregar
  escaneando un código QR o ingresando manualmente a la configuración del
  token. Red Hat lo mantiene bajo la licencia Apache 2.0 y es compatible
  con Android e iOS.

  \vspace{\baselineskip}
  \href{https://github.com/freeotp}{https://github.com/freeotp}
\end{frame}

\begin{frame}[c]{2FAS}

  \textbf{2FAS} es la forma más sencilla de habilitar la autenticación
  de dos factores para verificar su identidad e iniciar sesión de
  forma segura en cuentas con contraseñas de un solo uso basadas en el
  tiempo (TOTP) para mantener sus datos personales y contraseñas
  protegidos contra amenazas cibernéticas,
  todo desde una sola aplicación, 100\%. ¡gratis!

  \vspace{\baselineskip}
  \href{https://github.com/twofas}{https://github.com/twofas}

\end{frame}

\begin{frame}[c]{OATH Toolkit}

  \textbf{OATH Toolkit} proporciona componentes para crear sistemas de
  autenticación de contraseña de un solo uso.

  \vspace{\baselineskip}
  Contiene bibliotecas C compartidas, herramientas de línea de comandos y
  un módulo PAM. Las tecnologías admitidas incluyen el algoritmo HOTP
  basado en eventos (RFC 4226), el algoritmo TOTP basado en tiempo
  (RFC 6238) y el contenedor de claves simétricas portátil (PSKC, RFC 6030)
  para administrar datos de claves secretas.

  \vspace{\baselineskip}
  OATH significa Open AuTHentication, que es la organización que
  especifica los algoritmos.

  \vspace{\baselineskip}
  \href{https://gitlab.com/oath-toolkit/oath-toolkit}{https://gitlab.com/oath-toolkit/oath-toolkit}
\end{frame}

\begin{frame}[fragile]
  \frametitle{Ejemplo con OATH Toolkit}
  Generamos una llave, la ciframos y la guardamos
  \begin{lstlisting}[language=Bash,numbers=none]
$ echo "FSL - Vallarta 2024" | base32
IZJUYIBNEBLGC3DMMFZHIYJAGIYDENAK
$ echo "IZJUYIBNEBLGC3DMMFZHIYJAGIYDENAK" > $HOME/.2fa/fsl.key
$ gpg --symmetric --cipher-algo AES256 $HOME/.2fa/fsl.key
$ ls $HOME/.2fa/fsl*
fsl.key fsl.key.gpg
$ rm ~/.2fa/fsl.key
  \end{lstlisting}
\end{frame}

\begin{frame}[fragile]
  \frametitle{Ejemplo con OATH Toolkit}
  Desciframos y usamos nuestra llave con oathtool
  \begin{lstlisting}[language=Bash,numbers=none]
$ gpg --decrypt --quiet --cipher-algo AES256 \
  $HOME/.2fa/fsl.key.gpg | oathtool --base32 --totp -
030524
  \end{lstlisting}

  \pausa
  \vspace{\baselineskip}
  Si lo hacemos directo también sale
  \begin{lstlisting}[language=Bash,numbers=none]
$ oathtool --base32 --totp "IZJUYIBNEBLGC3DMMFZHIYJAGIYDENAK"
030524
$
  \end{lstlisting}
\end{frame}

\begin{frame}[c]{Aplicaciones que no son de código abierto}
  \begin{description}
    \item [Google Authenticator] genera verificación a 2 pasos para tu
      teléfono.
    \item [Microsoft Authenticator] una aplicación para verificar de forma
      rápida y segura su identidad en línea, para todas sus cuentas.
    \item [Authy] es una forma sencilla de administrar cuentas de
      autenticación de dos factores.
  \end{description}
\end{frame}
