% ex: ts=2 sw=2 sts=2 et filetype=tex
% SPDX-License-Identifier: CC-BY-SA-4.0

\section{Autenticación de múltiples factores (MFA)}

\begin{frame}[c]{La autenticación de múltiples factores (AMF)}
  \begin{itemize}
    \item Más comúnmente conocida por sus siglas en inglés \textbf{MFA}
      (\emph{Multi Factor Authentication})

    \pausa
    \item Es un método de \textbf{control de acceso informático} en el
      que a un usuario se le concede acceso al sistema solo después de que
      presente \textbf{dos o más pruebas diferentes} de que es quien dice ser.

    \pausa
    \item Estas pruebas pueden ser diversas, como una contraseña, que
      posea una \textbf{clave secundaria rotativa}, o un \textbf{certificado
      digital} instalado en el equipo, \emph{biometría}, entre otros.
  \end{itemize}
\end{frame}

\begin{frame}[c]{La autenticación de dos factores (A2F)}
  \begin{itemize}
    \item También usada la sigla inglesa \textbf{2FA} (\emph{de two-factor
      authentication})

    \pausa
    \item Es un \underline{método} que confirma que un \underline{usuario
      es quien dice ser} combinando dos componentes diferentes de entre:
      \begin{enumerate}
        \item algo que saben
        \item algo que tienen
        \item algo que son
      \end{enumerate}

    \pausa
    \item Es el \textbf{método más extendido en la actualidad} para acceder a
      cuentas de correo y otra aplicaciones, pero generalmente se solicita que
      el \textbf{usuario active voluntariamente esta capa de protección
      adicional}.
  \end{itemize}
\end{frame}

\begin{frame}[c]{Ejemplo de 2FA}
    \begin{columns}
        \column{0.5\textwidth}
        \begin{itemize}
          \item Un ejemplo de la vida cotidiana de este tipo de
            autenticación es la retirada de efectivo de un \textbf{cajero
            automático}.
          \pausa
          \item Solo tras combinar una tarjeta de crédito —\textbf{algo que
            el usuario posee}— y un pin —\textbf{algo que el usuario sabe}—
            se permite que la transacción se lleve a cabo.
        \end{itemize}
        \column{0.5\textwidth}
        \begin{center}
            \includegraphics[scale=0.5]{lfs/2fa-cajero.png}
        \end{center}
    \end{columns}
\end{frame}

\begin{frame}[c]{La autenticación de dos factores (A2F)}
  \begin{block}{}
      La \textbf{autenticación en dos pasos} o \textbf{verificación en
      dos pasos} es un método de confirmar la identidad de un usuario
      utilizando algo que conocen (contraseña) y un segundo factor distinto
      a lo que sean o posean.
  \end{block}

  \pausa
  \vspace{\baselineskip}
  Un ejemplo de un segundo paso es que el usuario tenga que introducir algo
  que le sea \textbf{enviado a través de un medio alternativo},
  o que tenga que introducir una serie de \textbf{dígitos generados por una
  aplicación} conocida por el usuario y el sistema de autenticación. 
\end{frame}

\begin{frame}[c]{Factores de autenticación}

  Los factores de autenticación de un patrón de autenticación de
  múltiples factores podría incluir:

  \vspace{\baselineskip}
  \begin{itemize}
    \item Algún objeto físico en posesión del usuario, como una memoria
      USB con un identificador único, una tarjeta de crédito, una llave, etc.

    \pausa
    \item Algún secreto conocido por el usuario, como una contraseña,
      un pin, etc.

    \pausa
    \item Alguna característica biométrica propia del usuario, como una
      huella dactilar, iris, voz, velocidad de escritura, patrón en los
      intervalos de pulsación de teclas, etc.
  \end{itemize}
\end{frame}

\section{Software para la autenticación a 2 pasos}

\begin{frame}[c]{}
  \vspace{\baselineskip}
\end{frame}

\begin{frame}[fragile]
  \frametitle{}
  \vspace{\baselineskip}
  \begin{lstlisting}[language=Bash,numbers=none]
  \end{lstlisting}
\end{frame}
