% ex: ts=2 sw=2 sts=2 et filetype=tex
% SPDX-License-Identifier: CC-BY-SA-4.0

\section{Cortafuegos "Firewalls" con UFW}

\begin{frame}[c]{Introducción}
  El kernel de Linux incluye el subsistema \textbf{Netfilter}, que se
  utiliza para manipular o decidir el destino del tráfico de red que se
  dirige hacia o a través de su servidor. Todas las soluciones modernas
  de cortafuegos de Linux utilizan este sistema para el filtrado de paquetes.
\end{frame}

\begin{frame}[c]{Introducción}
  El sistema de filtrado de paquetes del kernel sería de poca utilidad para
  los administradores sin una interfaz de espacio de usuario para
  administrarlo. Este es el propósito de \textbf{iptables}: cuando un paquete
  llega a su servidor, se entregará al subsistema \textbf{Netfilter} para su
  aceptación, manipulación o rechazo según las reglas proporcionadas desde
  el espacio de usuario a través de \emph{iptables}. Por lo tanto, iptables
  es todo lo que necesita para administrar su firewall, si está familiarizado
  con él, pero hay muchas interfaces disponibles para simplificar la tarea.
\end{frame}

\begin{frame}[c]{ufw - Cortafuegos sin complicaciones}
  La herramienta de configuración de firewall predeterminada para Ubuntu es
  \textbf{ufw}. Desarrollado para facilitar la configuración del cortafuegos
  de \emph{iptables}, \emph{ufw} proporciona una forma sencilla de crear un
  cortafuegos basado en host IPv4 o IPv6.

  \vspace{\baselineskip}
  ufw por defecto está inicialmente deshabilitado.
\end{frame}

\begin{frame}[c]{ufw - Cortafuegos sin complicaciones}
  De la página de manual de ufw:

  \vspace{\baselineskip}
  “ufw no pretende proporcionar una funcionalidad de firewall completa a
  través de su interfaz de comandos, sino que proporciona una manera fácil
  de agregar o eliminar reglas simples.

  \vspace{\baselineskip}
  Actualmente se utiliza principalmente para cortafuegos basados en host”.
\end{frame}

\begin{frame}[fragile]
  \frametitle{ufw - Cortafuegos sin complicaciones}
  Los siguientes son algunos ejemplos de cómo usar ufw:
  \begin{itemize}
    \item Primero, ufw debe estar habilitado. Desde un indicador de terminal,
      ingrese:
      \begin{lstlisting}[language=Bash]
sudo ufw enable
      \end{lstlisting}
    \item Para abrir un puerto (SSH en este ejemplo):
      \begin{lstlisting}[language=Bash]
sudo ufw allow ssh
      \end{lstlisting}
      UFW registra el significado del puerto \textbf{allow ssh} porque está 
      enumerado como servicio en el archivo \textbf{/etc/services}.

      Sin embargo, podemos escribir la regla equivalente especificando el
      puerto en vez del nombre del servicio. Por ejemplo, este comando
      funciona como el anterior:
      \begin{lstlisting}[language=Bash]
sudo ufw allow ssh
      \end{lstlisting}
  \end{itemize}
\end{frame}
