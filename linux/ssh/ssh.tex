% ex: ts=2 sw=2 sts=2 et filetype=tex
% SPDX-License-Identifier: CC-BY-SA-4.0
\documentclass[aspectratio=169]{beamer}

\usetheme{Boadilla}

\setbeamertemplate{navigation symbols}{} % To remove the navigation symbols from the bottom of all slides

\graphicspath{{../img/}}
\usepackage[utf8]{inputenc}
\usepackage[T1]{fontenc} %% https://tex.stackexchange.com/questions/664/why-should-i-use-usepackaget1fontenc
\usepackage{graphicx} % Allows including images
\usepackage{listings}
\usepackage{colortbl}

\title{Secure Shell} % The short title appears at the bottom of every slide,
                                               % the full title is only on the title page
\author{Miguel Bernal Marin} % Your name
\institute[Escuela] % Your institution as it will appear on the bottom of every slide, may be shorthand to save space
{
 Nombre de la Institución\\% Your institution for the title page
\medskip
\textit{correo} % Your email address
}
\date{
    \today
} % Date, can be changed to a custom date

%New colors defined below
\definecolor{codeNumbers}{rgb}{0.5,0.5,0.5}
\definecolor{editorGray}{rgb}{0.95, 0.95, 0.95}
\definecolor{editorOcher}{rgb}{1, 0.5, 0} % #FF7F00 -> rgb(239, 169, 0)
\definecolor{editorGreen}{rgb}{0, 0.5, 0} % #007C00 -> rgb(0, 124, 0)
\definecolor{brown}{rgb}{0.69,0.31,0.31}

%Code listing style named "mystyle"
\lstdefinestyle{mystyle}{
  % General design
  backgroundcolor=\color{editorGray},
  basicstyle={\ttfamily},
  numberstyle=\tiny\color{codeNumbers},
  xleftmargin={0.2cm},
  numbersep=5pt,
  numbers=left,
  stepnumber=1,
  firstnumber=1,
  numberfirstline=true,
  % Code design
  identifierstyle=\color{black},
  keywordstyle=\color{blue}\bfseries,
  ndkeywordstyle=\color{editorGreen}\bfseries,
  stringstyle=\color{editorOcher}\ttfamily,
  commentstyle=\color{brown}\ttfamily,
  % Code
  alsodigit={.:;},
  tabsize=2,
  showtabs=false,
  showspaces=false,
  showstringspaces=false,
  extendedchars=true,
  breaklines=true,
  % Español
  literate=%
  {á}{{\'a}}1 {é}{{\'e}}1 {í}{{\'i}}1 {ó}{{\'o}}1 {ú}{{\'u}}1
  {Á}{{\'A}}1 {É}{{\'E}}1 {Í}{{\'I}}1 {Ó}{{\'O}}1 {Ú}{{\'U}}1
  {¡}{{!`}}1  {¿}{{?`}}1
  {ñ}{{\~n}}1 {Ñ}{{\~N}}1
}

%"mystyle" code listing set
\lstset{style=mystyle}

%------------------------------------------------------------
%The next block of commands puts the table of contents at the
%beginning of each section and highlights the current section:
\AtBeginSection[]
{
  \begin{frame}
    \frametitle{Contenido}
    \tableofcontents[currentsection]
  \end{frame}
}
%------------------------------------------------------------

\newcommand{\pausa}{\pause} % Para usar una pausa en las presentaciones
%\newcommand{\pausa}{}      % Para que NO salgan las pausas

\begin{document}

\begin{frame}
    \titlepage
\end{frame}

\begin{frame}
    \frametitle{Contenido}
    \tableofcontents
\end{frame}

% ex: ts=2 sw=2 sts=2 et filetype=tex
% SPDX-License-Identifier: CC-BY-SA-4.0

\section{Introducción a JavaScript}

\begin{frame}[c]{¿Qué es JavaScript?}
  \textbf{JavaScript} (abreviado comúnmente \textbf{JS}) es un lenguaje de
  programación interpretado, dialecto del estándar ECMAScript. Se define
  como orientado a objetos, basado en prototipos, imperativo, débilmente
  tipado y dinámico. 

  \vspace{\baselineskip}
  Se utiliza principalmente del lado del cliente, implementado como parte
  de un navegador web permitiendo mejoras en la interfaz de usuario y
  páginas web dinámicas y JavaScript del lado del servidor (Server-side
  JavaScript o SSJS).
\end{frame}

\begin{frame}[c]{¿Qué es JavaScript?}

  JavaScript se diseñó con una sintaxis similar a C, aunque adopta nombres
  y convenciones del lenguaje de programación Java. Sin embargo, Java y
  JavaScript tienen semánticas y propósitos diferentes.

  \vspace{\baselineskip}
  Tradicionalmente se venía utilizando en páginas web HTML para realizar
  operaciones y únicamente en el marco de la aplicación cliente, sin acceso
  a funciones del servidor.

  \vspace{\baselineskip}
  Actualmente es ampliamente utilizado para enviar y recibir información del
  servidor junto con ayuda de otras tecnologías como \textbf{AJAX}.

\end{frame}

\begin{frame}[c]{¿Qué es AJAX?}

  \textbf{AJAX}, acrónimo de \textbf{A}synchronous \textbf{J}avaScript
  \textbf{A}nd \textbf{X}ML (JavaScript asíncrono y XML), es una técnica
  de desarrollo web para crear aplicaciones web asíncronas.

  \vspace{\baselineskip}
  Estas aplicaciones se ejecutan en el cliente, es decir, en el navegador
  de los usuarios mientras se mantiene la comunicación asíncrona con el
  servidor en segundo plano.

  \vspace{\baselineskip}
  De esta forma es posible interactuar con el
  servidor sin necesidad de recargar la página web, mejorando la
  interactividad, velocidad y usabilidad en las aplicaciones.
\end{frame}


\section{Características de JavaScript}

\begin{frame}[c]{Imperativo y estructurado}

  \begin{exampleblock}{}
    JavaScript es compatible con gran parte de la estructura de programación
    de C (por ejemplo, sentencias if, bucles for, sentencias switch, etc.).
  \end{exampleblock}

\end{frame}

\begin{frame}[c]{Dinámicos}

  \begin{block}{Tipado dinámico}
    El tipo de dato está asociado al valor, no a la variable. Por ejemplo,
    una variable x en un momento dado puede estar ligada a un número y más
    adelante, religada a una cadena de texto.
  \end{block}

  \begin{block}{Objetual}
    JavaScript está formado casi en su totalidad por objetos. Los objetos
    en JavaScript son arrays asociativos, mejorados con la inclusión de
    prototipos.

    \vspace{\baselineskip}
    Los nombres de las propiedades de los objetos son claves de tipo
    cadena: obj.x = 10 y obj['x'] = 10 son equivalentes.

    \vspace{\baselineskip}
    Las propiedades y sus valores pueden ser creados, cambiados o
    eliminados en tiempo de ejecución.
  \end{block}

\end{frame}

\begin{frame}[c]{Dinámicos (continuación)}

  \begin{block}{Evaluados en tiempo de ejecución}
    JavaScript incluye la función \textbf{eval} que permite evaluar
    expresiones expresadas como cadenas en tiempo de ejecución.
  \end{block}

\end{frame}

\begin{frame}[c]{Funcional}

  \begin{block}{Funciones de primera clase}
    A las funciones se les suele llamar ciudadanos de primera clase;
    son objetos en sí mismos. Como tal, poseen propiedades y métodos, 
    como .call() y .bind().

    \vspace{\baselineskip}
    Una función anidada es una función definida dentro de otra.
    Esta es creada cada vez que la función externa es invocada.
  \end{block}

\end{frame}

\begin{frame}[c]{Prototípico}

  \begin{block}{Prototipos}
    JavaScript usa prototipos en vez de clases para el uso de
    herencia. Es posible llegar a emular muchas de las
    características que proporcionan las clases en lenguajes orientados a
    objetos tradicionales por medio de prototipos en JavaScript.
  \end{block}

  \begin{block}{Funciones como constructores de objetos}
    Las funciones también se comportan como constructores. Prefijar una
    llamada a la función con la palabra clave new crear una nueva instancia
    de un prototipo, que heredan propiedades y métodos del constructor
    (incluidas las propiedades del prototipo de Object).
  \end{block}

\end{frame}

\begin{frame}[c]{Otras características}

  \begin{block}{Entorno de ejecución}
    JavaScript normalmente depende del entorno en el que se ejecute (por
    ejemplo, en un navegador web) para ofrecer objetos y métodos por los que
    los scripts pueden interactuar con el "mundo exterior".
  \end{block}

  \begin{block}{Funciones como métodos}
    A diferencia de muchos lenguajes orientados a objetos, no hay distinción
    entre la definición de función y la definición de método. Más bien, la
    distinción se produce durante la llamada a la función; una función puede
    ser llamada como un método.
  \end{block}

\end{frame}

\section{JavaScript en acción}

\begin{frame}[c]{Ejemplos de JavaScript}

  En seguida veremos algunos ejemplos de lo que se puede hacer en
  Javascript.

  \vspace{\baselineskip}
  Usa tu editor de texto favorito para copiar los ejemplos que aquí se
  mostrarán a continuación.
\end{frame}

\begin{frame}[c]{JavaScript puede cambiar el contenido HTML}

  Uno de los muchos métodos HTML de JavaScript es \textbf{getElementById()}.

  El siguiente ejemplo "busca" un elemento HTML (con id="demo") y cambia el
  contenido del elemento (innerHTML) a "Hola JavaScript":

  \vspace{\baselineskip}
  \begin{exampleblock}{Ejemplo:}
     document.getElementById("demo").innerHTML = "Hola JavaScript";
  \end{exampleblock}

\end{frame}

\begin{frame}[fragile]
  \frametitle{Ejemplo 1}
  \lstinputlisting{ejemplo01.html}
\end{frame}

\begin{frame}[c]{JavaScript puede cambiar el contenido HTML}

  \vspace{\baselineskip}
  \begin{alertblock}{Nota:}
    JavaScript acepta comillas simples o dobles para las cadenas de texto.
  \end{alertblock}

  \vspace{\baselineskip}
  \begin{exampleblock}{Ejemplo:}
     document.getElementById('demo').innerHTML = 'Hola JavaScript';
  \end{exampleblock}
\end{frame}

\begin{frame}[fragile]
  \frametitle{Ejemplo 2 (comillas)}
  \lstinputlisting{ejemplo02.html}
\end{frame}

\begin{frame}[fragile]
  \frametitle{JavaScript puede cambiar los valores de los atributos HTML}
  \lstinputlisting{ejemplo03.html}
\end{frame}

\begin{frame}[c]{JavaScript puede cambiar los estilos HTML (CSS)}

  Cambiar el estilo de un elemento HTML es una variante de cambiar un atributo HTML:

  \vspace{\baselineskip}
  \begin{exampleblock}{Ejemplo:}
    document.getElementById("demo").style.fontSize = "35px";
  \end{exampleblock}
\end{frame}

\begin{frame}[fragile]
  \frametitle{Ejemplo 4}
  \lstinputlisting{ejemplo04.html}
\end{frame}

\begin{frame}[c]{JavaScript puede ocultar elementos HTML}

  Se pueden ocultar elementos HTML cambiando el estilo de visualización:

  \vspace{\baselineskip}
  \begin{exampleblock}{Ejemplo:}
    document.getElementById("demo").style.display = "none"; 
  \end{exampleblock}
\end{frame}

\begin{frame}[fragile]
  \frametitle{Ejemplo 5}
  \lstinputlisting{ejemplo05.html}
\end{frame}

\begin{frame}[c]{JavaScript puede mostrar elementos HTML}

  También se pueden mostrar elementos HTML ocultos cambiando el estilo
  de visualización: 

  \vspace{\baselineskip}
  \begin{exampleblock}{Ejemplo:}
    document.getElementById("demo").style.display = "block";
  \end{exampleblock}
\end{frame}

\begin{frame}[fragile]
  \frametitle{Ejemplo 6}
  \lstinputlisting{ejemplo06.html}
\end{frame}


\end{document}
