% ex: ts=2 sw=2 sts=2 et filetype=tex
% SPDX-License-Identifier: CC-BY-SA-4.0

\section{GNU/Linux}

\begin{frame}[c]{¿Qué es GNU/Linux?}
    \begin{columns}
        \column{0.5\textwidth}
        \begin{center}
            \includegraphics[scale=0.5]{tux.png}
        \end{center}
        \pausa
        \column{0.5\textwidth}
        \textbf{GNU/Linux} es un sistema operativo (o una familia de sistemas
        operativos) tipo \textbf{Unix} compuesto por software libre y de código
        abierto. GNU/Linux surge de las contribuciones de varios proyectos
        de software, entre los cuales destacan GNU (iniciado por \emph{Richard
        Stallman} en 1983) y el kernel \textbf{Linux} (iniciado
        por \emph{Linus Torvalds} en 1991).
    \end{columns}
\end{frame}

\begin{frame}[c]{¿Qué es GNU/Linux?}
    \begin{columns}
        \column{0.5\textwidth}
        A pesar de que en la jerga cotidiana la mayoría de las personas usan
        el vocablo \textbf{Linux} para referirse a este sistema operativo, en
        realidad ese es solo el nombre del \textbf{kernel o núcleo}, ya que
        el sistema completo está formado también por una gran cantidad de
        componentes del proyecto GNU junto a componentes de terceros, que
        van desde compiladores hasta entornos de escritorio.
        \column{0.5\textwidth}
        \begin{center}
            \includegraphics[scale=0.5]{tux.png}
        \end{center}
    \end{columns}
\end{frame}

\begin{frame}[c]{¿Qué es GNU/Linux?}
    \begin{columns}
        \column{0.5\textwidth}
        \begin{center}
            \includegraphics[scale=0.5]{tux.png}
        \end{center}
        \column{0.5\textwidth}
        Cabe señalar que existen derivados que usan el núcleo Linux pero que
        no tienen componentes GNU, como por ejemplo el sistema operativo
        \textbf{Android}. También existen distribuciones de software GNU
        donde el núcleo Linux está ausente.
    \end{columns}
\end{frame}

\begin{frame}[c]{¿Qué es una distribución de Linux?}
    \begin{columns}
        \column{0.5\textwidth}
        \begin{center}
            \includegraphics[scale=0.4]{distros.png}
        \end{center}
        \pausa
        \column{0.5\textwidth}
        Una distribución (o \textbf{distro}) de Linux es una distribución
        de software basada en
        el núcleo Linux que incluye determinados paquetes de software para
        satisfacer las necesidades de un grupo específico de usuarios,
        dando así origen a ediciones domésticas, empresariales y para
        servidores.
    \end{columns}
\end{frame}

\begin{frame}[c]{Formatos de paquetes en Linux}

  \begin{block}{Los paquetes existentes en GNU/Linux}
    Son dependientes de la
    \textbf{distribución} en la que se estén usando; son usados comúnmente
    para la compresión de aplicaciones en diferentes formatos para
    distintos medios de instalación.
  \end{block}

  \vspace{\baselineskip}
  Estos son un conjunto de ficheros que contienen instrucciones para la
  reconstrucción de la aplicación dentro del sistema nuevo, dentro de estos,
  podemos encontrar, Paquetes Binarios y Paquetes de código Fuente.
\end{frame}

\begin{frame}[c]{Formatos de paquetes en Linux}
  \begin{description}
    \item [DEB] Contienen ejecutables, archivos de configuración,
      páginas de información, derechos de copyright y otras documentaciones,
      los paquetes \textbf{Debian} se colocan en archivos .deb.

      Estos paquetes también son usados por distribuciones basadas en la
      distribución Debian, algunas de estas, son: \textbf{Ubuntu, Kubuntu,
      ZorinOS, Linux Mint}, entre otras.

    \pausa
    \item [RPM] Por sus siglas en inglés \textbf{Redhat Package Manager},
      este tipo de empaquetado para Linux fue desarrollado para la
      distribución de Red Hat, con el fin de crear un sistema fácil de crear
      e instalar.

      Actualmente todas las distribuciones basadas en Red Hat ocupan los
      paquetes RPM, algunas de ellas son: \textbf{Fedora, CentOS y openSuSe},
      entre otras.
  \end{description}
\end{frame}

\begin{frame}[c]{Formatos de paquetes en Linux}
  \begin{description}
    \item [TGZ] Es un archivo de paquetes específico para Unix, comprimido
      con el compresor Gnu Zip. Es un paquete de código fuente, ocupado para
      contener aplicaciones, y su código fuente, para no tener que crear un
      tipo de paquete específico para cada distribución.

      A diferencia de los
      paquetes .deb, o .rpm, este no contiene instrucciones particulares de
      instalación para cada distribución, por lo que la instalación del
      contenido deberá ser compilado por el usuario.

    \pausa
    \item [Ebuild] Paquete usado solo por la distribución Gentoo, consiste en
      un script bash, ejecutable solo en un entorno específico. Sus archivos,
      deben de ser archivos de texto con la extensión .ebuild
  \end{description}
\end{frame}

\begin{frame}[c]{Formatos de paquetes en Linux}
  \begin{description}
    \item [Pacman] Combina un paquete binario simple, con un sistema de fácil
      construcción. La meta de pacman es el facilitamiento de la instalación
      de paquetes, sea que estén dentro de los repositorios oficiales de Arch,
      o creaciones de los mismos usuarios.
  \end{description}
\end{frame}

\begin{frame}[c]{Identifica el sistema de paqueteria de cada distro}
  \begin{center}
    \includegraphics[scale=0.5]{distros.png}
  \end{center}
\end{frame}
