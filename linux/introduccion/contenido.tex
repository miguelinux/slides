% ex: ts=2 sw=2 sts=2 et filetype=tex
% SPDX-License-Identifier: CC-BY-SA-4.0

\section{GNU/Linux}

\begin{frame}[c]{¿Qué es GNU/Linux?}
    \begin{columns}
        \column{0.5\textwidth}
        \begin{center}
            \includegraphics[scale=0.5]{tux.png}
        \end{center}
        \pausa
        \column{0.5\textwidth}
        \textbf{GNU/Linux} es un sistema operativo (o una familia de sistemas
        operativos) tipo \textbf{Unix} compuesto por software libre y de código
        abierto. GNU/Linux surge de las contribuciones de varios proyectos
        de software, entre los cuales destacan GNU (iniciado por \emph{Richard
        Stallman} en 1983) y el kernel \textbf{Linux} (iniciado
        por \emph{Linus Torvalds} en 1991).
    \end{columns}
\end{frame}

\begin{frame}[c]{¿Qué es GNU/Linux?}
    \begin{columns}
        \column{0.5\textwidth}
        A pesar de que en la jerga cotidiana la mayoría de las personas usan
        el vocablo \textbf{Linux} para referirse a este sistema operativo, en
        realidad ese es solo el nombre del \textbf{kernel o núcleo}, ya que
        el sistema completo está formado también por una gran cantidad de
        componentes del proyecto GNU junto a componentes de terceros, que
        van desde compiladores hasta entornos de escritorio.
        \column{0.5\textwidth}
        \begin{center}
            \includegraphics[scale=0.5]{tux.png}
        \end{center}
    \end{columns}
\end{frame}

\begin{frame}[c]{¿Qué es GNU/Linux?}
    \begin{columns}
        \column{0.5\textwidth}
        \begin{center}
            \includegraphics[scale=0.5]{tux.png}
        \end{center}
        \column{0.5\textwidth}
        Cabe señalar que existen derivados que usan el núcleo Linux pero que
        no tienen componentes GNU, como por ejemplo el sistema operativo
        \textbf{Android}. También existen distribuciones de software GNU
        donde el núcleo Linux está ausente.
    \end{columns}
\end{frame}

\begin{frame}[c]{¿Qué es una distribución de Linux?}
    \begin{columns}
        \column{0.5\textwidth}
        \begin{center}
            \includegraphics[scale=0.4]{distros.png}
        \end{center}
        \pausa
        \column{0.5\textwidth}
        Una distribución (o \textbf{distro}) de Linux es una distribución
        de software basada en
        el núcleo Linux que incluye determinados paquetes de software para
        satisfacer las necesidades de un grupo específico de usuarios,
        dando así origen a ediciones domésticas, empresariales y para
        servidores.
    \end{columns}
\end{frame}

\begin{frame}[c]{Formatos de paquetes en Linux}

  \begin{block}{Los paquetes existentes en GNU/Linux}
    Son dependientes de la
    \textbf{distribución} en la que se estén usando; son usados comúnmente
    para la compresión de aplicaciones en diferentes formatos para
    distintos medios de instalación.
  \end{block}

  \vspace{\baselineskip}
  Estos son un conjunto de ficheros que contienen instrucciones para la
  reconstrucción de la aplicación dentro del sistema nuevo, dentro de estos,
  podemos encontrar, Paquetes Binarios y Paquetes de código Fuente.
\end{frame}

\begin{frame}[c]{Formatos de paquetes en Linux}
  \begin{description}
    \item [DEB] Contienen ejecutables, archivos de configuración,
      páginas de información, derechos de copyright y otras documentaciones,
      los paquetes \textbf{Debian} se colocan en archivos .deb.

      Estos paquetes también son usados por distribuciones basadas en la
      distribución Debian, algunas de estas, son: \textbf{Ubuntu, Kubuntu,
      ZorinOS, Linux Mint}, entre otras.

    \pausa
    \item [RPM] Por sus siglas en inglés \textbf{Redhat Package Manager},
      este tipo de empaquetado para Linux fue desarrollado para la
      distribución de Red Hat, con el fin de crear un sistema fácil de crear
      e instalar.

      Actualmente todas las distribuciones basadas en Red Hat ocupan los
      paquetes RPM, algunas de ellas son: \textbf{Fedora, CentOS y openSuSe},
      entre otras.
  \end{description}
\end{frame}

\begin{frame}[c]{Formatos de paquetes en Linux}
  \begin{description}
    \item [TGZ] Es un archivo de paquetes específico para Unix, comprimido
      con el compresor Gnu Zip. Es un paquete de código fuente, ocupado para
      contener aplicaciones, y su código fuente, para no tener que crear un
      tipo de paquete específico para cada distribución.

      A diferencia de los
      paquetes .deb, o .rpm, este no contiene instrucciones particulares de
      instalación para cada distribución, por lo que la instalación del
      contenido deberá ser compilado por el usuario.

    \pausa
    \item [Ebuild] Paquete usado solo por la distribución Gentoo, consiste en
      un script bash, ejecutable solo en un entorno específico. Sus archivos,
      deben de ser archivos de texto con la extensión .ebuild
  \end{description}
\end{frame}

\begin{frame}[c]{Formatos de paquetes en Linux}
  \begin{description}
    \item [Pacman] Combina un paquete binario simple, con un sistema de fácil
      construcción. La meta de pacman es el facilitamiento de la instalación
      de paquetes, sea que estén dentro de los repositorios oficiales de Arch,
      o creaciones de los mismos usuarios.
  \end{description}
\end{frame}

\begin{frame}[c]{Identifica el sistema de paqueteria de cada distro}
  \begin{center}
    \includegraphics[scale=0.5]{distros.png}
  \end{center}
\end{frame}

\begin{frame}[c]{Entorno gráfico}
  Los sistemas operativos GNU/Linux pueden funcionar tanto en
  \textbf{entorno gráfico} como en \textbf{modo consola}, ya que el entorno
  gráfico no va explícitamente unido al resto de programas de manejo del
  sistema y puede usarse de forma opcional.

  \vspace{\baselineskip}
  La consola es común en distribuciones para servidores, mientras que la
  interfaz gráfica está orientada al usuario final, tanto de hogar como
  empresarial. Asimismo, también existen los \textbf{entornos de escritorio},
  que son un conjunto de programas formado por \emph{gestores de ventanas},
  iconos y muchas aplicaciones que facilitan la utilización de la
  computadora en modo gráfico.

\end{frame}

\begin{frame}[c]{Entorno gráfico}
  Los escritorios más populares en GNU/Linux son:

  \begin{itemize}
    \item GNOME
    \item KDE Plasma
    \item LXQt
    \item Xfce
    \item MATE
    \item Cinnamon
  \end{itemize}

  Aunque existen muchos más, también puede usarse con solo los
  \textbf{gestores de ventanas}, que son la columna vertebral de los entornos
  de escritorio, y los encargados de dibujar la interfaz de las aplicaciones
  y la composición.
\end{frame}

\section{Estructura de directorios}

\begin{frame}[c]{Estándar de jerarquía del sistema de archivos}
  El estándar de jerarquía del sistema de archivos (en inglés \textbf{Filesystem
  Hierarchy Standard}, también conocido por sus siglas \textbf{FHS}) es una
  norma que define los directorios principales y sus contenidos en el sistema
  operativo \textbf{GNU/Linux} y otros sistemas de la \emph{familia Unix}.

  \vspace{\baselineskip}
  Se diseñó originalmente en 1994 para estandarizar el sistema de archivos de
  las distribuciones de Linux, basándose en la tradicional organización de
  directorios de los sistemas Unix. En 1995 se amplió el ámbito del estándar
  a cualquier Unix que sea voluntariamente.
\end{frame}

\begin{frame}[c]{Estructura de los directorios}
  En \textbf{UNIX} y sistemas similares como BSD, GNU/Linux, todos los
  archivos y directorios aparecen bajo el directorio raíz, \textbf{/},
  aun cuando se encuentren en distintos dispositivos físicos.

  \vspace{\baselineskip}
  La mayoría de estos directorios existe en todos los sistemas operativos
  tipo UNIX, y generalmente son usados de igual forma, aunque no son
  consideradas obligatorias por otras plataformas GNU/Linux.
\end{frame}

\begin{frame}[c]{Estructura de los directorios}
	\begin{tikzpicture}
	[-,thick]
	\footnotesize
	\node {\texttt{/}} [edge from parent fork down]
		child {node {\texttt{bin}}}
		child {node {\texttt{etc}}}
		child {node {\texttt{home}}
			child {node {\texttt{anne}}}
			child {node {\texttt{sam}}}
			child {node {$\dots$}}
		}
		child {node {\texttt{lib}}}
		child {node {\texttt{opt}}}
		child {node {\texttt{proc}}}
		child {node {\texttt{tmp}}}
		child {node {\texttt{usr}}
			child {node {\texttt{bin}}
				child {node {\texttt{acyclic}}}
				child {node {\texttt{diff}}}
				child {node {\texttt{dot}}}
				child {node {\texttt{gc}}}
				child {node {\texttt{neato}}}
				child {node {$\dots$}}
			}
			child {node {\texttt{include}}}
			child {node {\texttt{local}}}
			child {node {\texttt{share}}}
			child {node {\texttt{src}}}
			child {node {$\dots$}}
		}
		child {node {$\dots$}};
	\end{tikzpicture}
\end{frame}

\begin{frame}[c]{Estructura general}
  \begin{description}
    \item [/] La raíz del sistema de archivos. Por lo general, se puede
			escribir, pero esto no es obligatorio.
    \pausa
		\item [/boot/] La partición de arranque utilizada para iniciar el sistema.
			En los sistemas EFI, esta es posiblemente la partición del sistema EFI
			(ESP).
    \pausa
		\item [/boot/efi/] Si la partición de arranque /boot/ se mantiene
			por separado de la partición del sistema EFI (ESP), esta última se monta
			aquí.
    \pausa
		\item [/etc/] Configuración específica del sistema. Este directorio puede
			o no ser de solo lectura.
    \pausa
		\item [/home/] La ubicación de los directorios de inicio del usuario
			normal. Posiblemente compartido con otros sistemas y nunca de solo
			lectura. Este directorio solo debe ser utilizado por usuarios normales,
			nunca por usuarios del sistema.
  \end{description}
\end{frame}

\begin{frame}[c]{Estructura general}
  \begin{description}
    \item [/root/] El directorio de inicio del usuario raíz. El directorio de
			inicio del usuario raíz se encuentra fuera de /home/para asegurarse de
			que el usuario raíz pueda iniciar sesión incluso sin /home/ estar
			disponible y montado.
    \pausa
    \item [/srv/] El lugar para almacenar información/archivos útiles del
			servidor en general, administrado por el administrador del sistema.
			No se imponen restricciones sobre cómo se organiza internamente este
			directorio. Generalmente se puede escribir y posiblemente se comparte
			entre sistemas.
    \pausa
    \item [/tmp/] El lugar para pequeños archivos temporales. Este directorio
			generalmente se monta como una instancia "tmpfs" y, por lo tanto, no
			debe usarse para archivos más grandes. (Use /var/tmp/ para archivos más
			grandes). Este directorio generalmente se vacía al arrancar. Además, los
			archivos a los que no se accede dentro de un tiempo determinado pueden
			eliminarse automáticamente.
  \end{description}
\end{frame}

\begin{frame}[c]{Datos de tiempo de ejecución}
  \begin{description}
    \item [/run/] Un sistema de archivos "tmpfs" para que los paquetes del
			sistema coloquen datos de tiempo de ejecución, archivos de socket y
			similares. Este directorio se vacía en el arranque y, por lo general,
			se puede escribir solo para programas privilegiados. Siempre se
			puede escribir.
    \pausa
    \item [/run/log/] Registros del sistema en tiempo de ejecución. Los
			componentes del sistema pueden colocar registros privados en este
			directorio. Siempre se puede escribir, incluso cuando /var/log/ es
			posible que aún no esté accesible.
    \pausa
    \item [/run/user/] Contiene directorios de tiempo de ejecución por
			usuario, cada uno de los cuales suele ser una instancia "tmpfs" montada
			individualmente. Siempre se puede escribir, vaciado en cada reinicio y
			cuando el usuario cierra la sesión.
  \end{description}
\end{frame}

\begin{frame}[c]{Recursos del sistema operativo proporcionados por el proveedor}
  \begin{description}
    \item [/usr/] Recursos del sistema operativo proporcionados por el
			proveedor. Por lo general, es de solo lectura, pero esto no es
			obligatorio. Posiblemente compartida entre varios hosts.
			El administrador no debe modificar este directorio, excepto al
			instalar o eliminar paquetes proporcionados por proveedores.
    \pausa
    \item [/usr/bin/] Binarios y ejecutables para comandos de usuario que
			aparecerán en la \$PATH ruta de búsqueda. Se recomienda no colocar
			binarios en este directorio que no sean útiles para la invocación
			desde un shell (como los binarios de demonios); estos deben colocarse en
			un subdirectorio de /usr/lib/ en su lugar.
    \pausa
    \item [/usr/include/] Archivos de encabezado de la API de C y C++ de
      las bibliotecas del sistema.
  \end{description}
\end{frame}

\begin{frame}[c]{Recursos del sistema operativo proporcionados por el proveedor}
  \begin{description}
    \item [/usr/lib/] Datos estáticos de proveedores privados compatibles con
      todas las arquitecturas (aunque no necesariamente independientes de la
      arquitectura). Tenga en cuenta que esto incluye ejecutables internos u
      otros archivos binarios que no se invocan regularmente desde un shell.
      Dichos binarios pueden ser para cualquier arquitectura compatible con
      el sistema.
    \pausa
    \item [/usr/lib/arch-id/] Ubicación para colocar bibliotecas dinámicas,
      también llamada \$libdir. El identificador de arquitectura a utilizar
      se define en la lista de especificadores de arquitectura
      multiarquitectura (tuplas) . Las ubicaciones heredadas de \$libdir
      son /usr/lib/, /usr/lib64/. Este directorio no debe usarse para datos
      específicos del paquete, a menos que estos datos también dependan de la
      arquitectura.
  \end{description}
\end{frame}

\begin{frame}[c]{Recursos del sistema operativo proporcionados por el proveedor}
  \begin{description}
    \item [/usr/share/] Recursos compartidos entre varios paquetes, como
      documentación, páginas man, información de zona horaria, fuentes y otros
      recursos. Por lo general, la ubicación precisa y el formato de los
      archivos almacenados debajo de este directorio están sujetos a
      especificaciones que garantizan la interoperabilidad.
    \pausa
    \item [/usr/share/doc/] Documentación del sistema operativo o paquetes
      del sistema
  \end{description}
\end{frame}

\begin{frame}[c]{Datos del sistema de variables persistentes}
  \begin{description}
    \item [/var/] Datos persistentes y variables del sistema. Se puede escribir
      durante el funcionamiento normal del sistema. Es posible que este
      directorio se complete previamente con datos proporcionados por el
      proveedor, pero las aplicaciones deberían poder reconstruir los archivos
      y directorios necesarios en esta subjerarquía en caso de que falten, ya
      que el sistema podría iniciarse sin que se complete este directorio.
    \pausa
    \item [/var/lib/] Datos persistentes del sistema. Los componentes del
      sistema pueden colocar datos privados en este directorio
  \end{description}
\end{frame}

\begin{frame}[c]{Datos del sistema de variables persistentes}
  \begin{description}
    \item [/var/log/] Registros persistentes del sistema. Los componentes del
      sistema pueden colocar registros privados en este directorio, aunque se
      recomienda realizar la mayoría de los registros a través de las
      llamadas \textbf{syslog}(3)  y \textbf{sd\_journal\_print}(3).
    \pausa
    \item [/var/tmp/] El lugar para archivos temporales más grandes y
      persistentes. A diferencia de /tmp/, este directorio generalmente se
      monta desde un sistema de archivos físico persistente y, por lo tanto,
      puede aceptar archivos más grandes.
  \end{description}
\end{frame}

\begin{frame}[c]{Kernel virtual y sistemas de archivos API}
  \begin{description}
    \item [/dev/] El directorio raíz para los nodos de dispositivos. Por lo
      general, este directorio se monta como una instancia "devtmpfs", pero
      puede ser de un tipo diferente en configuraciones de espacio
      aislado/contenedores. Este directorio es administrado conjuntamente por
      el kernel y systemd-udevd(8) , y otros componentes no deben escribir
      en él.
    \pausa
    \item [/proc/] Un sistema de archivos de kernel virtual que expone la
      lista de procesos y otras funciones. Este sistema de archivos es
      principalmente una API para interactuar con el kernel y no un lugar
      donde se puedan almacenar archivos normales.
  \end{description}
\end{frame}

\begin{frame}[c]{Kernel virtual y sistemas de archivos API}
  \begin{description}
    \item [/proc/sys/] Una jerarquía debajo /proc/ que expone una cantidad de
      parámetros ajustables del kernel. La forma principal de configurar los
      ajustes en este árbol de archivos API es a través de los archivos
      \textbf{sysctl.d}(5).
    \pausa
    \item [/sys/] Un sistema de archivos de kernel virtual que expone los
      dispositivos descubiertos y otras funciones. Este sistema de archivos es
      principalmente una API para interactuar con el kernel y no un lugar
      donde se puedan almacenar archivos normales.
  \end{description}
\end{frame}

\begin{frame}[c]{Enlaces simbólicos de compatibilidad}
  \begin{description}
    \item [/bin/]
    \item [/sbin/]
    \item [/usr/sbin/] Estos enlaces simbólicos de compatibilidad apuntan a
      /usr/bin/, lo que garantiza que los scripts y los binarios que hacen
      referencia a estas rutas heredadas encuentren correctamente sus binarios.
    \pausa
    \item [/lib/] Este enlace simbólico de compatibilidad apunta a
      /usr/lib/, lo que garantiza que los programas que hacen referencia a
      esta ruta heredada encuentren correctamente sus recursos.
    \pausa
    \item [/var/run/] Este enlace simbólico de compatibilidad apunta a /run/,
      lo que garantiza que los programas que hacen referencia a esta ruta
      heredada encuentren correctamente sus datos de tiempo de ejecución.
  \end{description}
\end{frame}
