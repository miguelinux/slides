% ex: ts=2 sw=2 sts=2 et filetype=tex
% SPDX-License-Identifier: CC-BY-SA-4.0

\section{GNU/Linux}

\begin{frame}[c]{¿Qué es GNU/Linux?}
    \begin{columns}
        \column{0.5\textwidth}
        \textbf{GNU/Linux} es un sistema operativo (o una familia de sistemas
        operativos) tipo \textbf{Unix} compuesto por software libre y de código
        abierto. GNU/Linux surge de las contribuciones de varios proyectos
        de software, entre los cuales destacan GNU (iniciado por \emph{Richard
        Stallman} en 1983) y el kernel \textbf{Linux} (iniciado
        por \emph{Linus Torvalds} en 1991).
        \column{0.5\textwidth}
        \begin{center}
            \includegraphics[scale=0.5]{tux.png}
        \end{center}
    \end{columns}
\end{frame}

\begin{frame}[c]{¿Qué es GNU/Linux?}
    \begin{columns}
        \column{0.5\textwidth}
        \begin{center}
            \includegraphics[scale=0.5]{tux.png}
        \end{center}
        \column{0.5\textwidth}
        A pesar de que en la jerga cotidiana la mayoría de las personas usan
        el vocablo \textbf{Linux} para referirse a este sistema operativo, en
        realidad ese es solo el nombre del \textbf{kernel o núcleo}, ya que
        el sistema completo está formado también por una gran cantidad de
        componentes del proyecto GNU junto a componentes de terceros, que
        van desde compiladores hasta entornos de escritorio.
    \end{columns}
\end{frame}

\begin{frame}[c]{¿Qué es GNU/Linux?}
    \begin{columns}
        \column{0.5\textwidth}
        Cabe señalar que existen derivados que usan el núcleo Linux pero que
        no tienen componentes GNU, como por ejemplo el sistema operativo
        \textbf{Android}. También existen distribuciones de software GNU
        donde el núcleo Linux está ausente.
        \column{0.5\textwidth}
        \begin{center}
            \includegraphics[scale=0.5]{tux.png}
        \end{center}
    \end{columns}
\end{frame}
