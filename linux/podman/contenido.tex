% ex: ts=2 sw=2 sts=2 et filetype=tex
% SPDX-License-Identifier: CC-BY-SA-4.0

\section{Podman y Quadlets}

\begin{frame}[c]{Podman (Pod Manager)}
    \begin{columns}
        \column{0.4\textwidth}
        \begin{center}
            \includegraphics[scale=0.4]{lfs/podman.png}
        \end{center}
        \column{0.6\textwidth}
        \begin{itemize}
          \item Es una herramienta sencilla y \textbf{sin demonio}.
          \pausa
          \item Un \textbf{motor de contenedores} con todas las funciones.
          \pausa
          \item Proporciona una línea de comandos comparable a
            \textbf{Docker-CLI}
            que facilita la transición desde otros motores de contenedores
            y permite la gestión de \textbf{pods, contenedores e imágenes}.
        \end{itemize}
    \end{columns}
\end{frame}

\begin{frame}[c]{Quadlets}
  \begin{itemize}
    \item Está disponible a partir de Podman v4.4.
    \pausa
    \item Se \textbf{describe} cómo ejecutar un contenedor en un formato
      muy similar al de los archivos de unidad de \textbf{systemd}.
    \pausa
    \item Las descripciones del contenedor se centran en los
      \textbf{detalles relevantes} del contenedor y \underline{ocultan}
      los detalles técnicos de su ejecución en \textbf{systemd}.
  \end{itemize}
\end{frame}

\begin{frame}[c]{Directorios}

  Los archivos de quadlets para el usuario \textbf{root} se pueden
  colocar en los siguientes directorios, ordenados por prioridad.

  \vspace{\baselineskip}
  Esto significa que los quadlets con nombre duplicado que se
  encuentren en \textbf{/run} tienen prioridad sobre los de
  \textbf{/etc} y \textbf{/usr}:
  \pausa
  \begin{alertblock}{Cuadlets temporales, generalmente usados para pruebas:}
    \begin{itemize}
      \item /run/containers/systemd/
    \end{itemize}
  \end{alertblock}
  \pausa
  \begin{exampleblock}{Cuadlets definidos por el administrador del sistema:}
    \begin{itemize}
      \item /etc/containers/systemd/
    \end{itemize}
  \end{exampleblock}
  \pausa
  \begin{block}{Cuadlets definidos por la distribución:}
    \begin{itemize}
      \item /usr/share/containers/systemd/
    \end{itemize}
  \end{block}
\end{frame}

\begin{frame}[c]{Directorios}

  Los archivos de quadlets para los usuario \textbf{no-root} se pueden
  colocar en los siguientes directorios:

  \vspace{\baselineskip}
  \begin{exampleblock}{Usuario no-root:}
    \begin{itemize}
      \item \$XDG\_RUNTIME\_DIR/containers/systemd/
      \pausa
      \item \$XDG\_CONFIG\_HOME/containers/systemd/
      \pausa
        \begin{itemize}
          \item \$HOME/.config/containers/systemd/
        \end{itemize}
      \pausa
      \item /etc/containers/systemd/users/\$(UID)
      \pausa
      \item /etc/containers/systemd/users/
    \end{itemize}
  \end{exampleblock}
\end{frame}

\section{Unidades de systemd}

\begin{frame}[c]{Container units [Container]}
\end{frame}

\begin{frame}[c]{Container units [Container]}
\end{frame}

\begin{frame}[c]{Container units [Container]}
\end{frame}
