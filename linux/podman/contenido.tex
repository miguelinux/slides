% ex: ts=2 sw=2 sts=2 et filetype=tex
% SPDX-License-Identifier: CC-BY-SA-4.0

\section{Podman y Quadlets}

\begin{frame}[c]{Podman (Pod Manager)}
    \begin{columns}
        \column{0.4\textwidth}
        \begin{center}
            \includegraphics[scale=0.4]{lfs/podman.png}
        \end{center}
        \column{0.6\textwidth}
        \begin{itemize}
          \item Es una herramienta sencilla y \textbf{sin demonio}.
          \pausa
          \item Un \textbf{motor de contenedores} con todas las funciones.
          \pausa
          \item Proporciona una línea de comandos comparable a
            \textbf{Docker-CLI}
            que facilita la transición desde otros motores de contenedores
            y permite la gestión de \textbf{pods, contenedores e imágenes}.
        \end{itemize}
    \end{columns}
\end{frame}

\begin{frame}[c]{Quadlets}
  \begin{itemize}
    \item Está disponible a partir de Podman v4.4.
    \pausa
    \item Se \textbf{describe} cómo ejecutar un contenedor en un formato
      muy similar al de los archivos de unidad de \textbf{systemd}.
    \pausa
    \item Las descripciones del contenedor se centran en los
      \textbf{detalles relevantes} del contenedor y \underline{ocultan}
      los detalles técnicos de su ejecución en \textbf{systemd}.
  \end{itemize}
\end{frame}

\begin{frame}[c]{Directorios}

  Los archivos de quadlets para el usuario \textbf{root} se pueden
  colocar en los siguientes directorios, ordenados por prioridad.

  Esto significa que los quadlets con nombre duplicado que se
  encuentren en \textbf{/run} tienen prioridad sobre los de
  \textbf{/etc} y \textbf{/usr}:
  \pausa
  \begin{alertblock}{Cuadlets temporales, generalmente usados para pruebas:}
    \begin{itemize}
      \item /run/containers/systemd/
    \end{itemize}
  \end{alertblock}
  \pausa
  \begin{exampleblock}{Cuadlets definidos por el administrador del sistema:}
    \begin{itemize}
      \item /etc/containers/systemd/
    \end{itemize}
  \end{exampleblock}
  \pausa
  \begin{block}{Cuadlets definidos por la distribución:}
    \begin{itemize}
      \item /usr/share/containers/systemd/
    \end{itemize}
  \end{block}
\end{frame}

\begin{frame}[c]{Directorios}

  Los archivos de quadlets para los usuario \textbf{no-root} se pueden
  colocar en los siguientes directorios:

  \vspace{\baselineskip}
  \begin{exampleblock}{Usuario no-root:}
    \begin{itemize}
      \item \$XDG\_RUNTIME\_DIR/containers/systemd/
      \pausa
      \item \$XDG\_CONFIG\_HOME/containers/systemd/
      \pausa
        \begin{itemize}
          \item \$HOME/.config/containers/systemd/
        \end{itemize}
      \pausa
      \item /etc/containers/systemd/users/\$(UID)
      \pausa
      \item /etc/containers/systemd/users/
    \end{itemize}
  \end{exampleblock}
\end{frame}

\section{Unidades de systemd}

\begin{frame}[c]{Generador de Systemd}
  \begin{itemize}
    \item Podman permite la \textbf{creación e inicio} de contenedores
          (y la creación de volúmenes) a través de systemd
          mediante un \textbf{generador systemd}.
    \pausa
    \item Estos archivos se leen durante el arranque (y al ejecutar
      \textbf{systemctl daemon-reload}) y generan los archivos de
      unidad de \textbf{servicio systemd} correspondientes.
    \pausa
    \item El generador Podman lee las rutas de búsqueda anteriores y
      lee archivos con las extensiones
      \begin{itemize}
        \item .container
        \pausa
        \item .volume
        \pausa
        \item .network
        \pausa
        \item .build
        \pausa
        \item .pod
        \pausa
        \item .kube
      \end{itemize}
      y para cada archivo genera un archivo \textbf{.service} con el
      mismo nombre.
  \end{itemize}
\end{frame}

\begin{frame}[c]{Generador de Systemd}
  \begin{itemize}
    \item Los archivos de unidad generados se pueden iniciar y
          gestionar con \textbf{systemctl} como cualquier otro
          servicio systemd.
    \pausa
    \item \textbf{systemctl {--user} list-unit-files} enumera
          los archivos de unidades existentes en el sistema.
  \end{itemize}
\end{frame}

\begin{frame}[c]{Container units [Container]}

  Se utilizan para administrar contenedores mediante la ejecución del
  comando \textbf{podman run}.

  \vspace{\baselineskip}
  \begin{itemize}
    \item Extensión de archivo: \textbf{.container}
    \item Nombre de la sección: \textbf{[Container]}
    \item Campos requeridos: \textbf{Image} que describe la imagen del
      contenedor que ejecuta el servicio.
  \end{itemize}
\end{frame}

\begin{frame}[c]{Kube units [Kube]}
  Se utilizan para administrar contenedores definidos en archivos
  \textbf{YAML} de \textbf{Kubernetes} mediante la ejecución del
  comando \textbf{podman kube play}.
  \begin{itemize}
    \item Extensión de archivo: \textbf{.kube}
    \item Nombre de la sección: \textbf{[Kube]}
    \item Campos requeridos: \textbf{Yaml} que define la ruta al
      archivo YAML de \textbf{Kubernetes}.
  \end{itemize}
\end{frame}

\begin{frame}[c]{Network units [Network]}
  Se utilizan para crear \textbf{redes Podman} que pueden referenciarse en
  archivos .container o .kube.
  \begin{itemize}
    \item Extensión de archivo: \textbf{.network}
    \item Nombre de la sección: \textbf{[Network]}
    \item Campos requeridos: Ninguno
  \end{itemize}
\end{frame}

\begin{frame}[c]{Volume units [Volume]}
  Se utilizan para crear \textbf{volúmenes Podman} que pueden
  referenciarse en archivos .container.
  \begin{itemize}
    \item Extensión de archivo: \textbf{.volume}
    \item Nombre de la sección: \textbf{[Volume]}
    \item Campos requeridos: Ninguno
  \end{itemize}
\end{frame}

\begin{frame}[c]{Build units [Build]}
  Se utilizan para describir el comando para \textbf{construir}
  una imagen.
  \begin{itemize}
    \item Extensión de archivo: \textbf{.build}
    \item Nombre de la sección: \textbf{[Build]}
    \item Campos requeridos: \textbf{ImageTag} y cualquiera de
      \textbf{File} o \textbf{SetWorkingDirectory}
  \end{itemize}
\end{frame}

\begin{frame}[c]{Pod units [Pod]}
  Se utilizan para crear pods mediante la ejecución del
  comando \textbf{podman pod create}.
  \begin{itemize}
    \item Extensión de archivo: \textbf{.pod}
    \item Nombre de la sección: \textbf{[Pod]}
    \item Campos requeridos: Ninguno
  \end{itemize}
\end{frame}

\section{Ejemplo 1: Contenedor mínimo}

