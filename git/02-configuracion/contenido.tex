% ex: ts=2 sw=2 sts=2 et filetype=tex
% SPDX-License-Identifier: CC-BY-SA-4.0

\section{Nombre y correo}

\begin{frame}[c]{Antes de comenzar ...}
  \begin{itemize}
    \item Si estas en \textbf{Windows} utiliza el programa de \textbf{Git
          Bash}
    \pause
    \item Si estas en \textbf{Mac OS}  o \textbf{Linux} abre una terminal de
          sistema.
    \pause
    \item ¿Dónde se ejecutan estos comandos?
    \pause
      \begin{itemize}
        \item Los comandos se ejecuten dentro del directorio que contiene el
              \textbf{repositorio}
      \end{itemize}
    \pause
    \item ¿Cómo sé que es el directorio del repositorio?
    \pause
      \begin{itemize}
        \item El directorio del repositorio contiene un directorio llamado
              \textbf{.git}
      \end{itemize}
  \end{itemize}
\end{frame}

\begin{frame}[fragile]
  \frametitle{Comandos básicos de linea de comandos}
  \begin{lstlisting}
ls
  \end{lstlisting}
  Lista los archivos contenidos en el directorio/carpeta actual
  \pause
  \begin{lstlisting}
pwd
  \end{lstlisting}
  Muestra la ruta completa del directorio actual
  \pause
  \begin{lstlisting}
cd DESTINO
  \end{lstlisting}
  Se cambia al directorio DESTINO
  \pause
  \begin{lstlisting}
mkdir CARPETA
  \end{lstlisting}
  Crea el directorio llamado CARPETA
\end{frame}

\begin{frame}[fragile]
  \frametitle{Configurando tu nombre en Git}
  \begin{lstlisting}
git config user.name
  \end{lstlisting}
  Muestra el nombre configurado
  \pause
  \begin{lstlisting}
git config --global user.name "Nombre Completo"
  \end{lstlisting}
  Asigna el nombre
\end{frame}

\begin{frame}[fragile]
  \frametitle{Configurando tu correo electrónico en Git}
  \begin{lstlisting}
git config user.email
  \end{lstlisting}
  Muestra el correo electrónico configurado
  \pause
  \begin{lstlisting}
git config --global user.email usuario@servidor.com
  \end{lstlisting}
  Asigna el correo electrónico
\end{frame}

\section{Editor de texto}

\begin{frame}[fragile]
  \frametitle{Configurando el editor de texto en Git}
  El editor de texto sirve para crear los "\textbf{commits}".
  Por default \emph{git} utiliza el editor \textbf{vim}
  \pause
  \begin{lstlisting}
git config core.editor
  \end{lstlisting}
  Muestra el editor configurado
  \pause
  \begin{lstlisting}
git config --global core.editor EDITOR
  \end{lstlisting}
  Asigna el editor de texto a usar
  \begin{itemize}
    \item En \textbf{Windows} el EDITOR puede ser \textbf{notepad}
    \item En \textbf{Mac OS} y en \textbf{Linux} el EDITOR puede
          ser \textbf{nano}
  \end{itemize}
\end{frame}

\section{Llaves SSH en GitHub}

\begin{frame}[c]{¿Qué son las llaves SSH?}
  \begin{itemize}
    \item Las llaves SSH se utilizan para autentificarse en un servidor
          remoto.
    \pause
    \item \textbf{Github} las utiliza para autentificarse en su servidor
          y poder actualizar el contenido de un repositorio.
  \end{itemize}
\end{frame}

\begin{frame}[fragile]
  \frametitle{Llave/Clave privada y publica}

  Para generar una nueva llave privada y publica se hace con el siguiente
  comando
  \vspace{\baselineskip}
  \begin{lstlisting}[language=Bash]
ssh-keygen -t ed25519 -C "un comentario o tu correo"
  \end{lstlisting}
  \vspace{\baselineskip}
  Y para fines escolares le dan \textbf{3 veces} la tecla \textbf{Enter}
  
  \begin{alertblock}{Nota:}
    Si vas a usar un sistema "viejo" que no soporte el algoritmo Ed25519 usa
    el siguiente comando:

    \begin{lstlisting}[language=Bash]
ssh-keygen -t rsa -b 4096 -C "un comentario o tu correo"
    \end{lstlisting}
  \end{alertblock}
\end{frame}
