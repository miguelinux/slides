% ex: ts=2 sw=2 sts=2 et filetype=tex
% SPDX-License-Identifier: CC-BY-SA-4.0

\section{Introducción a Python}

\begin{frame}[c]{¿Qué es Python?}
    \begin{columns}
        \column{0.5\textwidth}
        Python es un lenguaje de programación \textit{interpretado} cuya
        filosofía hace hincapié en la \underline{legibilidad de su código}.
        Se trata de un lenguaje de programación \textbf{multiparadigma}, ya
        que soporta parcialmente la \textit{orientación a objetos,
        programación imperativa} y, en menor medida, \textit{programación
        funcional}. \\~\\

        Es un lenguaje interpretado, dinámico y multiplataforma. 
        \column{0.5\textwidth}
        \begin{center}
            \includegraphics[scale=0.5]{python-logo.png}
        \end{center}
    \end{columns}
\end{frame}

\begin{frame}[c]{Historia}
  \begin{columns}
    \column{0.3\textwidth}
        \begin{center}
            \includegraphics[scale=0.25]{Guido_van_Rossum_OSCON_2006.jpg}
        \end{center}
    \column{0.7\textwidth}
    \begin{itemize}
      \item Creado por Guido Van Rossum
      \pausa
      \item Holanda principios de la decade de los 90's
      \pausa
      \item Sintaxis simple, práctica e intuitiva
      \pausa
      \item De proposito general: NASA, motor de busqueda de Google, Bolsa de
        Valores de Nueva York
      \pausa
      \item Interpretado
      \pausa
      \item Multiparadigma
      \pausa
      \item Tipado dinámico
      \pausa
      \item Python 2.x ya esta \textbf{descontinuado}
      \pausa
      \item Solo hay que usar versiones de Python 3.x
    \end{itemize}
  \end{columns}
\end{frame}

\begin{frame}[c]{¿Qué puede hacer Python?}
  Python puede:
  \begin{itemize}
    \item ser usado en un servidor para crear aplicaciones web
    \pausa
    \item ser usado junto algún software para crear flojos de trajo.
    \pausa
    \item conectarse a un sistema de base de datos. También puede leer y
          modificar archivos.
    \pausa
    \item ser usado para manejar una gran cantidad de datos y funciones
          matemáticas complejas.
    \pausa
    \item ser usado para el desarrollo de prototipado rápido o para software
          listo para producción.
  \end{itemize}
\end{frame}

\begin{frame}[c]{¿Por qué Python?}
  \begin{itemize}
    \item Python trabaja en diferentes plataformas (Windows, Mac, Linux,
      Raspberry Pi, etc).
    \pausa
    \item Python tiene una sintaxis simple, similar a la del idioma Ingles.
    \pausa
    \item Python tiene una sintaxis que permite a los desarrolladores
      escribir programas con pocas líneas de código en comparación con otros
      lenguajes de programación.
    \pausa
    \item Python es un lenguaje de programación interpretado, esto quiere
      decir que el código puede ser ejecutado tan pronto como el código sea
      escrito. En otras palabras se pueden hacer programas prototipos muy
      rápido.
    \pausa
    \item Python puede ser tratado de forma \textbf{procedural},
      \textbf{orientado a objetos} o \textbf{funcional}
  \end{itemize}
\end{frame}

\begin{frame}[c]{La sintaxis de Python comparada con otros lenguajes}
  \begin{itemize}
    \item Python fue diseñado para facilitar la lectura y tiene algunas
      similitudes con el idioma inglés con influencia de las matemáticas.
    \pausa
    \item Python usa \emph{salto de líneas} para completar un comando, a
      diferencia de otros lenguajes de programación que a menudo usan punto
      y coma o paréntesis.
    \pausa
    \item Python se basa en la \emph{sangría}, usando espacios en blanco, para
      definir el alcance; como el alcance de los bucles, funciones y clases.
      Otros lenguajes de programación a menudo usan corchetes para este
      propósito.
  \end{itemize}
\end{frame}

\begin{frame}[c]{Ejecución de programas}
  \begin{block}{Código fuente}
    El código generado en Lenguaje Python se almacena en un archivo con
    extensión \textbf{.py}
  \end{block}
\end{frame}

\section{Reglas de codificación}

\begin{frame}[c]{Reglas de codificación}
  \begin{itemize}
    \item Los comentario de una línea inician con \textcolor{codeComment}{\#}
    \pausa
    \item Los comentarios de varias líneas inician con
    \textcolor{codeComment}{"""} y terminan con \textcolor{codeComment}{"""}
    \pausa
    \item La sangría (sangrado o \textit{indentado}) cuenta
    \pausa
    \item No poner signos de puntuación al final de la instrucción
    \pausa
    \item Es sensible a mayúsculas
    \pausa
    \item Errores de programación:
          \begin{itemize}
            \item Sintaxis
            \item Ejecución
            \item Lógicos
          \end{itemize}
  \end{itemize}
\end{frame}

\begin{frame}[fragile]
  \frametitle{Comentarios en Python}
  \vspace{\baselineskip}
  Los comentarios pueden ser usados para

  \begin{itemize}
    \item explicar el código de Python.
    \item hacer el código mas legible.
    \item prevenir la ejecución de código de pruebas.
  \end{itemize}

  \pausa
  \begin{lstlisting}[language=Python]
# Este es un comentario
# y otro comentario
print("Hola Mundo")

print("Hola Mundo") # Este es un comentario
# print("Código de prueba")

"""
Este es un comentario
escrito en mas
de una sola linea
"""
  \end{lstlisting}
\end{frame}

\begin{frame}[fragile]
  \frametitle{Sangría de Python}

  \vspace{\baselineskip}
  La \textbf{sangría} (indentado) se refiere a los espacios al comienzo
  de una línea de código.

  \vspace{\baselineskip}
  Mientras que en otros lenguajes de programación la sangría en el código
  es solo para legibilidad, \underline{la sangría en Python es muy
  importante}.

  \vspace{\baselineskip}
  Python usa sangría para indicar un bloque de código.
  \begin{lstlisting}[language=Python]
  if 5 > 2:
      print("Cinco es más grande que dos!")
  \end{lstlisting}

  Python le dará un error si omite la sangría:
  \begin{lstlisting}[language=Python]
  if 5 > 2:
  print("Cinco es más grande que dos!")
  \end{lstlisting}

\end{frame}


\begin{frame}[fragile]
  \frametitle{Sangría de Python}

  \vspace{\baselineskip}
  El número de espacios depende de usted como programador,
  pero debe ser al menos uno.

  \vspace{\baselineskip}
  \begin{lstlisting}[language=Python]
  if 5 > 2:
   print("Cinco es más grande que dos!")
  if 5 > 2:
        print("Cinco es más grande que dos!")
  \end{lstlisting}

  Tienes que usar la misma cantidad de espacios en el mismo bloque
  de código, de lo contrario Python te dará un error:

  \vspace{\baselineskip}
  \begin{lstlisting}[language=Python]
  if 5 > 2:
   print("Cinco es más grande que dos!")
        print("Cinco es más grande que dos!")
  \end{lstlisting}
\end{frame}


\section{Variables}

\begin{frame}[fragile]
  \frametitle{Variables en Python}

  Las variables son contenedores para almacenar valores de datos.

  \begin{block}{Creando variables}
    Python no tiene un comando para declarar una variable. \\~\\
    Una variable es \textbf{creada} en el momento que se le asigna un
    valor por primera vez.
  \end{block}

  \begin{lstlisting}[language=Python]
x = 5
y = "Juan"
print(x)
print(y)
  \end{lstlisting}
\end{frame}

\begin{frame}[fragile]
  \frametitle{"Casting" en Python}

  Si se quiere especificar el tipo de dato de una variable, esto se puede hacer
  haciendo "\textbf{casting}"

  \vspace{\baselineskip}
  \begin{lstlisting}[language=Python]
  x = str(3)   # x guarda '3'
  y = int(3)   # y guarda 3
  z = float(3) # z guarda 3.0
  \end{lstlisting}
\end{frame}

\begin{frame}[fragile]
  \frametitle{Obteniendo el tipo de dato}

  Se puede obtener el tipo de dato con la función
  \textcolor{codeKeyword2}{type}()

  \vspace{\baselineskip}
  \begin{lstlisting}[language=Python]
  x = str(3)   # x guarda '3'
  y = int(3)   # y guarda 3
  z = float(3) # z guarda 3.0

  print( type(x) )
  print( type(y) )
  print( type(z) )
  \end{lstlisting}
\end{frame}

\begin{frame}[fragile]
  \frametitle{Nombre de variables}

  \vspace{\baselineskip}
  Las variables pueden tener un nombre corto (como \textit{x} y \textit{y})
  o tener un nombre más \textbf{descriptivo} (\textit{edad}, \textit{nombre},
  \textit{volumen\_total}). Las reglas para las variables en Python son:
  \pausa

  \vspace{\baselineskip}
  \begin{itemize}
    \item El nombre de la variable tiene que \textbf{comenzar} con una
      \textbf{letra} o un \textbf{guión bajo} (\_).
    \pausa
    \item El nombre de una variable \textbf{no} puede iniciar con
      \textit{número.}
    \pausa
    \item El nombre de una variable solo puede contener caracteres
      alfa-numéricos y guiones bajo (\textit{A-Z, a-z, 0-9 y "\_"})
    \pausa
    \item Los nombre de las variables son sensibles a mayúsculas y minúsculas
      (edad, Edad y EDAD son tres variables deferentes)
    \pausa
  \item No se pueden usar las vocales acentuadas (á,é,í,ó,ú), ni la letra ñ
  \end{itemize}
\end{frame}

\begin{frame}[fragile]
  \frametitle{Nombre de variables}

  Estos son unos ejemplos de nombre de variables

  \vspace{\baselineskip}
  \begin{lstlisting}[language=Python]
 mivar = "Ana"
 mi_var = "Juan"
 _mi_var = "Pedro"
 miVar = "Lucia"
 MIVAR = "Maria"
 mivar2 = "Estela"
 mi_variable = "Mateo"
 anio_de_nacimento = 2010
 el_nombre_de_mi_variable = "Miguel"
  \end{lstlisting}
\end{frame}

\begin{frame}[c]{Variables con múltiples palabras}
  Un nombre de una variable con múltiples palabras puede ser difícil de leer.
  Hay varias técnicas que se pueden usar para que sean mas legibles.
  \pausa
  \begin{block}{Camel case}
    Cada palabra, excepto la primera, comienza con una mayúscula:
    \textbf{elNombreDeMiVariable}
  \end{block}
  \pausa
  \begin{block}{Pascal case}
    Cada palabra comienza con una mayúscula:
    \textbf{ElNombreDeMiVariable}
  \end{block}
  \pausa
  \begin{exampleblock}{Snake case}
    Cada palabra esta separada por un guión bajo:
    \textbf{el\_nombre\_de\_mi\_variable}
  \end{exampleblock}
\end{frame}

\begin{frame}[fragile]
  \frametitle{Asignando múltiples variables}

  Python permite asignar valores a múltiples variables en una sola linea:

  \vspace{\baselineskip}
  \begin{lstlisting}[language=Python]
  color, sabor, olor = "azul", "dulce", "rosas"
  print(color)
  print(sabor)
  print(olor)
  \end{lstlisting}

  \begin{alertblock}{Nota}
    Hay que asegurar que el número de variables sea igual al número de valores
    asignados, ya que de lo contrario se podría generar un error.
  \end{alertblock}
\end{frame}

\begin{frame}[fragile]
  \frametitle{Un valor para múltiples variables}

  También se puede asignar el mismo valor a múltiples variables en una sola
  línea.

  \vspace{\baselineskip}
  \begin{lstlisting}[language=Python]
  x = y = z = "Manzana"
  print(x)
  print(y)
  print(z)
  \end{lstlisting}
\end{frame}

\begin{frame}[fragile]
  \frametitle{Vaciar una Colección}

  Si se tiene una colección de valores en una \textit{lista}, \textit{tupla},
  etc. Python permite extraer los valores en variables separadas. A esto
  se le llama en ingles \textbf{unpacking}

  \vspace{\baselineskip}
  \begin{lstlisting}[language=Python]
  frutas = ["manzana", "platano", "melon"]
  a, b, c = frutas
  print(a)
  print(b)
  print(c)
  \end{lstlisting}
\end{frame}

\begin{frame}[fragile]
  \frametitle{Salida de variables}

  La sentencia \textcolor{codeKeyword}{print} de Python es usada para
  imprimir en pantalla las variables.

  \vspace{\baselineskip}
  Para combinar un texto y una variable (de texto), Python usa el carácter
  \textbf{+}

  \begin{lstlisting}[language=Python]
x = "increíble"
print("Python es " + x)
  \end{lstlisting}

  \pausa
  Para combinar cualquier tipo de variable se usa una coma "\textbf{,}" y
  esta forma de salida agrega un espacio en blanco entre la salida de cada
  variable.

  \vspace{\baselineskip}
  \begin{lstlisting}[language=Python]
x = "increíble"
print("Python es", x)
  \end{lstlisting}


\end{frame}

\begin{frame}[fragile]
  \frametitle{Salida de variables}

  Para números, el carácter de \textbf{+} sirve como un
  operador matemático:

  \vspace{\baselineskip}
  \begin{lstlisting}[language=Python]
x = 5
y = 10
print(x + y)
  \end{lstlisting}
\end{frame}

\begin{frame}[fragile]
  \frametitle{Salida de variables}

  \begin{alertblock}{}
    Si se combina una cadena de texto con un número, Python dará un error.
  \end{alertblock}

  \begin{lstlisting}[language=Python]
x = 5
y = "Juan"
print(x + y)
  \end{lstlisting}

  \pausa
  \begin{exampleblock}{}
    Se puede usar la coma para evitar que nos de error.
  \end{exampleblock}

  \begin{lstlisting}[language=Python]
x = 5
y = "Juan"
print(x, y)
  \end{lstlisting}
\end{frame}

\begin{frame}[fragile]
  \frametitle{Variables globales}

  Las variables que son creadas fuera de una función (como se muestra en el
  ejemplo de abajo) son conocidas como \emph{variables globales}.

  \vspace{\baselineskip}
  Las variables globales puede ser usadas donde sea, dentro y fuera de una
  función.

  \vspace{\baselineskip}
  \begin{lstlisting}[language=Python]
x = "increíble"

def unaFuncion():
    print("Python es " + x)

unaFuncion()
  \end{lstlisting}
\end{frame}

\begin{frame}[fragile]
  \frametitle{Variables globales}

  Si se crea una variable con el mismo nombre dentro de una función,
  esta variable será \emph{local} y solo se puede usar dentro de la función.
  La variable global con el mismo nombre permanecerá como estaba, global y
  con el valor original.

  \vspace{\baselineskip}
  \begin{lstlisting}[language=Python]
x = "increíble"

def unaFuncion():
    x = "fantástico"
    print("Python es " + x)

unaFuncion()

print("Python es " + x)
  \end{lstlisting}
\end{frame}

\begin{frame}[fragile]
  \frametitle{La palabra reservada \textbf{global}}

  Normalmente, cuando se crea una variable dentro de una función, esa
  variable es local y solo se puede usar dentro de esa función.

  \vspace{\baselineskip}
  Para crear una variable global dentro de una función, puede utilizar
  la palabra reservada \textcolor{codeKeyword}{global}.

  \vspace{\baselineskip}
  \begin{lstlisting}[language=Python]
def unaFuncion():
    global x
    x = "fantástico"

unaFuncion()

print("Python es " + x)
  \end{lstlisting}
\end{frame}

\begin{frame}[fragile]
  \frametitle{La palabra reservada \textbf{global}}

  Además, use la palabra reservada \textcolor{codeKeyword}{global} si desea
  cambiar una variable global dentro de una función.

  \vspace{\baselineskip}
  \begin{lstlisting}[language=Python]
x = "increíble"

def unaFuncion():
    global x
    x = "fantástico"

unaFuncion()

print("Python es " + x)
  \end{lstlisting}
\end{frame}
