% ex: ts=2 sw=2 sts=2 et filetype=tex
% SPDX-License-Identifier: CC-BY-SA-4.0

\section{Módulos en Python}

\begin{frame}[c]{¿Qué es un módulo?}
  \begin{itemize}
    \item Considere que un módulo es lo mismo que un código de una biblioteca.
    \item Un archivo que contiene un conjunto de funciones que desea incluir
      en su aplicación.
  \end{itemize}
\end{frame}

\begin{frame}[fragile]
  \frametitle{Crear un módulo}
  Para crear un módulo simplemente guarde el código que desea en un
  archivo con la extensión de archivo \textbf{.py}:

  \begin{exampleblock}{Ejemplo:}
  Guarde este código en un archivo llamado \textbf{mimodulo.py}
    \begin{lstlisting}[language=Python]
def saludo(nombre):
  print("Hola, " + nombre)
    \end{lstlisting}
  \end{exampleblock}
\end{frame}

\begin{frame}[fragile]
  \frametitle{Utilizar un módulo}
  Ahora podemos utilizar el módulo que acabamos de crear,
  usando la declaración \textbf{import}:

  \begin{exampleblock}{Ejemplo:}
  Importa el módulo llamado \textbf{mimodulo} y llama a la función de saludo:
    \begin{lstlisting}[language=Python]
import mimodulo

mimodulo.saludo("Maria")
    \end{lstlisting}
  \end{exampleblock}

  \begin{block}{Nota:}
    Al utilizar una función de un módulo, utilice la sintaxis:
    \textbf{nombre\_de\_modulo.nombre\_de\_función}.
  \end{block}
\end{frame}

\begin{frame}[fragile]
  \frametitle{Variables en el módulo}

  El módulo puede contener funciones, como ya se ha descrito,
  pero también variables de todo tipo (listas, diccionarios, objetos, etc.):

  \begin{exampleblock}{Ejemplo:}
    Guarda este código en el archivo \textbf{mimodulo.py}
    \begin{lstlisting}[language=Python]
persona1 = {
  "nombre": "Juan",
  "edad": 23,
  "país": "México"
}
    \end{lstlisting}
  \end{exampleblock}
\end{frame}

\begin{frame}[fragile]
  \frametitle{Variables en el módulo}

  \begin{exampleblock}{Ejemplo:}
    Importa el módulo llamado \textbf{mimodulo} y acceda al diccionario
    \textbf{persona1}:
    \begin{lstlisting}[language=Python]
import mimodulo

edad = mimodulo.persona1["edad"]

print(edad)
    \end{lstlisting}
  \end{exampleblock}
\end{frame}

\begin{frame}[fragile]
  \frametitle{Cambiar el nombre de un módulo}

  \begin{block}{Nombrar un módulo:}
    Puedes nombrar el archivo del módulo como quieras,
    pero debe tener la extensión de archivo \textbf{.py}
  \end{block}
  \pausa
  Puedes crear un alias al importar un módulo,
  utilizando la palabra clave \textbf{as}:

  \begin{exampleblock}{Ejemplo:}
    Crear un alias para mimodulo llamado mm:
    \begin{lstlisting}[language=Python]
import mimodulo as mm

nombre = mm.persona1["nombre"]

print(nombre)
    \end{lstlisting}
  \end{exampleblock}
\end{frame}

\begin{frame}[fragile]
  \frametitle{Módulos integrados}
  Hay varios módulos integrados en Python que puedes importar cuando quieras.

  \begin{exampleblock}{Ejemplo:}
    Importar y utilizar el módulo math:
    \begin{lstlisting}[language=Python]
import math

raiz = math.sqrt(9)
print(raiz)
    \end{lstlisting}
  \end{exampleblock}
\end{frame}

\begin{frame}[fragile]
  \frametitle{Usando la función dir()}
  Hay una función integrada para listar todos los nombres de
  funciones (o variables) de un módulo. La función \textbf{dir()}:
  \begin{exampleblock}{Ejemplo:}
    Enumere todos los nombres definidos que pertenecen al módulo de math:
    \begin{lstlisting}[language=Python]
import math

lista = dir(math)
print(lista)
    \end{lstlisting}
  \end{exampleblock}
  \begin{block}{Nota:}
    La función dir() se puede utilizar en todos los módulos,
    también en los que uno mismo crea.
  \end{block}
\end{frame}

\begin{frame}[fragile]
  \frametitle{Importar desde el módulo}
  Puede elegir importar solo partes de un módulo,
  utilizando la palabra clave \textbf{from}.
  \begin{exampleblock}{Ejemplo:}
    El módulo nombrado \textbf{mimodulo} tiene una función y un diccionario:
    \begin{lstlisting}[language=Python]
def saludo(nombre):
  print("Hola, " + nombre)

persona1 = {
  "nombre": "Juan",
  "edad": 23,
  "país": "México"
}
    \end{lstlisting}
  \end{exampleblock}
\end{frame}

\begin{frame}[fragile]
  \frametitle{Importar desde el módulo}

  \begin{exampleblock}{Ejemplo:}
    Importamos únicamente el diccionario persona1 del módulo:
    \begin{lstlisting}[language=Python]
from mimodulo import persona1

print(persona1["nombre"])
    \end{lstlisting}
  \end{exampleblock}
  \begin{alertblock}{Nota:}
    Al importar con la palabra clave \textbf{from}, no utilizamos
    el nombre del módulo al referirse a los elementos del módulo.
    Ejemplo: \textbf{persona1["nombre"]}, y no mimodulo.persona1["nombre"]
  \end{alertblock}
\end{frame}
