% vi: ts=2 sw=2 sts=2 et filetype=tex
% SPDX-License-Identifier: CC-BY-SA-4.0

\section{Linter}

\begin{frame}[c]{What is a Linter?}
  \begin{definition}
    A linter is a software tool that \textbf{analyzes source code} for errors,
    style issues, and vulnerabilities. Its goal is to improve code quality.
  \end{definition}

  \vspace{\baselineskip}
  The name comes from the original tool called \textbf{Lint}, which was
  designed to check C code for errors and inefficiencies.
\end{frame}

\begin{frame}[c]{Linter functions}
  \begin{itemize}
    \item Identify syntax errors, logical errors, typos, and
      potential security vulnerabilities
    \pausa
    \item Comply with coding standards and style guidelines
    \pausa
    \item Suggest or apply corrections
    \pausa
    \item Format code
    \pausa
    \item Enforce coding standards.
  \end{itemize}
\end{frame}

\begin{frame}[c]{Importance of lintering}
  \begin{itemize}
    \item Identify errors before they become real problems at runtime
    \pausa
    \item Maintain code more consistently
    \pausa
    \item Simplify tasks
    \pausa
    \item Ensure code quality.
  \end{itemize}
\end{frame}

\section{Linter Python tools}

\begin{frame}[c]{Ruff}
  \begin{itemize}
    \item An extremely fast Python linter and code formatter,
      written in Rust.
    \item \href{https://docs.astral.sh/ruff/}{https://docs.astral.sh/ruff/}
    \pausa
    \item Ruff aims to be orders of magnitude faster than alternative tools
      while integrating more functionality behind a single, common interface.
    \pausa
    \item Ruff can be used to replace Flake8 (plus dozens of plugins), Black,
      isort, pydocstyle, pyupgrade, autoflake, and more, all while executing
      tens or hundreds of times faster than any individual tool.
  \end{itemize}
\end{frame}

\begin{frame}[c]{Black}
  \begin{itemize}
    \item The uncompromising code formatter
    \item \href{https://black.readthedocs.io/en/stable/}{https://black.readthedocs.io/en/stable/}
    \pausa
    \item Black gives you speed, determinism, and freedom from pycodestyle
      nagging about formatting
    \pausa
    \item Black makes code review faster by producing the smallest diffs
      possible. 
    \pausa
    \item Blackened code looks the same regardless of the project you’re
      reading.
  \end{itemize}
\end{frame}

\begin{frame}[c]{Bandit}
  \begin{itemize}
    \item Bandit is a tool designed to find common security issues in
      Python code. 
    \item \href{https://bandit.readthedocs.io/en/latest/}{https://bandit.readthedocs.io/en/latest/}
  \end{itemize}
\end{frame}

\begin{frame}[c]{Reorder Python imports}
  \begin{itemize}
    \item Tool for automatically reordering python imports
    \item \href{https://github.com/asottile/reorder-python-imports}{https://github.com/asottile/reorder-python-imports}
    \pausa
    \item Similar to isort but uses static analysis more.
  \end{itemize}
\end{frame}

\section{Pre-commit Tool}

\begin{frame}[c]{Pre-commit}
  \begin{itemize}
    \item A framework for managing and maintaining multi-language pre-commit
      hooks.
    \item \href{https://pre-commit.com/}{https://pre-commit.com/}
    \pausa
    \item Git hook scripts are useful for identifying simple issues before
      submission to code review
    \pausa
    \item We run our hooks on every commit to automatically point out issues
      in code such as missing semicolons, trailing whitespace, and debug
      statements.
  \end{itemize}
\end{frame}

\begin{frame}[fragile]
  \frametitle{Pre-commit installation}

  Using pip:
  
  \begin{lstlisting}[language=Bash]
 pip install pre-commit
  \end{lstlisting}
  \vspace{\baselineskip}

  RPM based Linux:
  
  \begin{lstlisting}[language=Bash]
 dnf install pre-commit
  \end{lstlisting}
  \vspace{\baselineskip}

  DEB based Linux:
  
  \begin{lstlisting}[language=Bash]
 apt install pre-commit
  \end{lstlisting}
\end{frame}

\begin{frame}[fragile]
  \frametitle{Pre-commit configuration file}
  \begin{itemize}
    \item create a file named \textbf{.pre-commit-config.yaml}
    \pausa
    \item you can generate a very basic configuration using
      \textbf{pre-commit sample-config}
      or copy from our repo ;-)
  \end{itemize}

  \begin{lstlisting}[language=Bash]
repos:
  - repo: https://github.com/pre-commit/pre-commit-hooks
    rev: v5.0.0
    hooks:
      - id: check-merge-conflict
      - id: debug-statements
      - id: fix-byte-order-marker
      - id: trailing-whitespace
      - id: end-of-file-fixer
      - id: check-yaml
      - id: check-added-large-files
      - id: detect-private-key
  \end{lstlisting}
\end{frame}

\begin{frame}[fragile]
  \frametitle{Install the git hook scripts}
  \begin{itemize}
    \item run \textbf{pre-commit install} to set up the git hook scripts
  \end{itemize}

  \begin{lstlisting}[language=Bash]
$ pre-commit install
pre-commit installed at .git/hooks/pre-commit
$
  \end{lstlisting}

  \begin{itemize}
    \item now \textbf{pre-commit} will run automatically on \textbf{git
      commit}!
  \end{itemize}
\end{frame}

\begin{frame}[fragile]
  \frametitle{Run against all the files (optional)}
  \begin{itemize}
    \item it's usually a good idea to run the hooks against all of the
      files when adding new hooks 
    \item usually \textbf{pre-commit} will only run on the changed files
      during git hooks
  \end{itemize}

  %\begin{lstlisting}[language=Bash, basicstyle=\tiny]
\begin{tiny}
\begin{verbatim}
$ pre-commit run --all-files
[INFO] Initializing environment for https://github.com/pre-commit/pre-commit-hooks.
[INFO] Initializing environment for https://github.com/psf/black.
[INFO] Installing environment for https://github.com/pre-commit/pre-commit-hooks.
[INFO] Once installed this environment will be reused.
[INFO] This may take a few minutes...
[INFO] Installing environment for https://github.com/psf/black.
[INFO] Once installed this environment will be reused.
[INFO] This may take a few minutes...
Check Yaml...............................................................Passed
Fix End of Files.........................................................Passed
Trim Trailing Whitespace.................................................Failed
- hook id: trailing-whitespace
- exit code: 1

Files were modified by this hook. Additional output:

Fixing sample.py

black....................................................................Passed
\end{verbatim}
\end{tiny}
  %\end{lstlisting}
\end{frame}
