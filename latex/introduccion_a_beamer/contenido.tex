% ex: ts=2 sw=2 sts=2 et filetype=tex
% SPDX-License-Identifier: CC-BY-SA-4.0

\section{\LaTeX{} y Beamer}

\begin{frame}[c]{¿Qué es \LaTeX{}?}
  \LaTeX{} (escrito LaTeX en texto sin formato) es un sistema de composición
  de textos orientado a la \textbf{creación de documentos} escritos que
  presenten una alta calidad tipográfica.

  \vspace{\baselineskip}
  Por sus características y posibilidades, se usa de forma especialmente
  intensa en la generación de \textbf{artículos y libros científicos} que
  incluyen, entre otros elementos, \textbf{expresiones matemáticas}.
\end{frame}

\begin{frame}[c]{¿Qué es Beamer?}
  \textbf{Beamer} es una clase de LaTeX para la creación de presentaciones.

  \vspace{\baselineskip}
  Funciona con \textbf{pdflatex}, \textbf{dvips} y \textbf{LyX}.
  El nombre viene del vocablo alemán "beamer", un pseudo-anglicismo que
  significa videoproyector.

  \vspace{\baselineskip}
  Al estar basado en LaTeX, Beamer es especialmente útil para preparar
  presentaciones en las que es necesario mostrar gran cantidad de expresiones
  matemáticas, el fuerte de dicho sistema de maquetación.
\end{frame}

\begin{frame}[fragile]
  \frametitle{Ejemplo de Beamer}
  \begin{lstlisting}[language={[LaTeX]TeX}]
\documentclass{beamer}
\title{Titulo}
\author{Anónimo}
\institute{FSL}
\date{2021}

\begin{document}

\frame{\titlepage}

\begin{frame}[c]{Titulo}
  Texto
\end{frame }

\end{document}
  \end{lstlisting}
\end{frame}


\begin{frame}[c]{Lista de códigos}
  \textbf{listados (listings)} es usado para componer listados de código
  fuente usando LaTeX

  \vspace{\baselineskip}
  El paquete permite al usuario componer programas (código de programación)
  dentro de LaTeX; TeX lee el código fuente directamente; no se necesita
  ningún procesador frontal.

  \vspace{\baselineskip}
  Las palabras clave, los comentarios y las cadenas se pueden componer
  usando diferentes estilos.
\end{frame}

\begin{frame}[fragile]
  \frametitle{Ejemplo de listings}
  \begin{lstlisting}[language={[LaTeX]TeX}]

\begin{frame}[fragile]
  \frametitle{Ejemplo de listings}
  \begin{lstlisting}[language=C]
#include <stdio.h>

int main (int argc, char *argv[])
{
  printf("Hola mundo\n");
  return 0;
}
  \end{ lstlisting }
\end{ frame }


  \end{lstlisting}
\end{frame}

\begin{frame}[fragile]
  \frametitle{Ejemplo de listings}
  \begin{lstlisting}[language=C]
#include <stdio.h>

int main (int argc, char *argv[])
{
  printf("Hola mundo\n");
  return 0;
}
  \end{lstlisting}
\end{frame}

\section{\LaTeX{} y Git}

\begin{frame}[c]{Git}
  \textbf{Git} es un software de control de versiones diseñado por
  \texttt{Linus Torvalds},
  pensando en la eficiencia, la confiabilidad y compatibilidad del
  mantenimiento de versiones de aplicaciones cuando estas tienen un
  gran número de archivos de código fuente.

  \vspace{\baselineskip}
  Su propósito es llevar registro de los cambios en archivos de computadora
  incluyendo coordinar el trabajo que varias personas realizan sobre archivos
  compartidos en un repositorio de código.
\end{frame}


\begin{frame}[fragile]
  \frametitle{Divide y venceras}
  Separar el contenido de la portada/configuración de la presentación.
  \begin{lstlisting}[language=Bash,numbers=none]
$ tree ssh
ssh/
|-- contenido.tex
|-- ssh.tex
$
  \end{lstlisting}
\end{frame}

\begin{frame}[fragile]
  \frametitle{Divide y venceras}
  Crea diferentes branches para cada una de las versiones/estilos de las
  presentaciones.
  \begin{lstlisting}[language=Bash,numbers=none]
$ git branch
estilo1
estilo2
main
oscuro1
oscuro2
$
  \end{lstlisting}
\end{frame}

\begin{frame}[c]{Usando diferentes estilos}
  \includegraphics[height=3.8cm]{s1v1.png}
  \includegraphics[height=3.8cm]{s1v2.png}
  \includegraphics[height=3.8cm]{s1v3.png}
  \includegraphics[height=3.8cm]{s1v4.png}
\end{frame}

\begin{frame}[c]{Git Large File Storage (LFS)}
  LFS es una extensión Git de código abierto para versionar archivos
  grandes.

  \vspace{\baselineskip}
  Git Large File Storage (LFS) reemplaza archivos grandes, como muestras
  de audio, videos, conjuntos de datos y gráficos con punteros de texto
  dentro de Git, mientras almacena el contenido del archivo en un servidor
  remoto como GitHub.com o GitHub Enterprise.
\end{frame}

\begin{frame}[fragile]
  \frametitle{Git LFS}
  El archivo de atributos para el git LFS es:
  \begin{lstlisting}[language=Bash,numbers=none]
$ cat .gitattributes
*.png filter=lfs diff=lfs merge=lfs -text
*.jpg filter=lfs diff=lfs merge=lfs -text
$
  \end{lstlisting}
\end{frame}

\section{Aplicaciones de presentación }

\begin{frame}[c]{pdfpc}
  \textbf{pdfpc} Una consola de presentador con soporte multimonitor
  para archivos PDF

  \vspace{\baselineskip}
  pdfpc es un visor de presentaciones basado en GTK que utiliza una salida
  de múltiples monitores similar a Keynote para proporcionar metainformación
  al orador durante la presentación.

  \vspace{\baselineskip}
  Es capaz de mostrar una ventana de presentación normal en una pantalla
  mientras muestra una descripción general más sofisticada en la otra,
  proporcionando información como una imagen de la siguiente diapositiva,
  así como el tiempo restante de la presentación.

  \vspace{\baselineskip}
  pdfpc procesa documentos PDF, que se pueden crear utilizando casi todos
  los programas de presentación modernos.
\end{frame}

\begin{frame}[c]{Pympress}
  \textbf{Pympress} es una herramienta de presentación en PDF diseñada
  para configuraciones de pantalla dual, como presentaciones y charlas
  públicas. Altamente configurable, con todas las funciones y portable.

  \vspace{\baselineskip}
  Viene con muchas características excelentes:

  \begin{itemize}
    \item Admite gifs integrados (listos para usar), videos y audios
      (con integración VLC o Gstreamer)
    \item anotaciones de texto mostradas en la ventana del presentador
    \item Admite de forma nativa las notas del proyector en la segunda
      pantalla, ¡así como las páginas de notas de Libreoffice!
  \end{itemize}

  \vspace{\baselineskip}
  Pympress es un software gratuito, distribuido bajo los términos de la
  licencia GPL (versión 2 o posterior).
\end{frame}
