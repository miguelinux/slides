% ex: ts=2 sw=2 sts=2 et filetype=tex
% SPDX-License-Identifier: CC-BY-SA-4.0

\section{Elementos de HTML}

\begin{frame}[c]{Elementos HTML}

  Un elemento HTML se define mediante una \textbf{etiqueta} de inicio,
  algo de contenido y una \textbf{etiqueta} de finalización:

  \vspace{\baselineskip}
  \textcolor{blue}{<nombre\_de\_la\_etiqueta>} el contenido va aquí
  \textcolor{blue}{</nombre\_de\_la\_etiqueta>}

  \pausa
  \vspace{\baselineskip}
  El elemento HTML es todo, desde la etiqueta de inicio hasta la
  etiqueta final:

  \vspace{\baselineskip}
  \textcolor{blue}{<h1>}Mi primer encabezado\textcolor{blue}{</h1>}

  \vspace{\baselineskip}
  \textcolor{blue}{<p>}Mi primer párrafo.\textcolor{blue}{</p>}

  \begin{exampleblock}{Nota:}
    La etiqueta final siempre tiene una / al inicio.
  \end{exampleblock}
\end{frame}


\begin{frame}[c]{Elementos HTML anidados}

  Los elementos HTML se pueden anidar (esto significa que los
  elementos pueden contener otros elementos).

  \vspace{\baselineskip}
  Todos los documentos HTML constan de elementos HTML anidados.

  \vspace{\baselineskip}
  El siguiente ejemplo contiene cuatro elementos HTML (<html>, <body>,
  <h1> y <p>)
\end{frame}

\begin{frame}[fragile]
  \frametitle{Ejemplo}

  \begin{lstlisting}
<!DOCTYPE html>
<html>
<body>

<h1>Mi primer encabezado</h1>
<p>Mi primer párrafo.</p>

</body>
</html>
  \end{lstlisting}

  El elemento <html> es el elemento raíz y define todo el documento HTML.
  Tiene una etiqueta de inicio <html> y una etiqueta de fin </html>.

  \vspace{\baselineskip}
  Luego, dentro del elemento <html> hay un elemento <body>.
  El elemento <body> define el cuerpo del documento.
  Tiene una etiqueta de inicio <body> y una etiqueta de fin </body>.

\end{frame}

\begin{frame}[fragile]
  \frametitle{Ejemplo}

  Luego, dentro del elemento <body> hay otros dos elementos: <h1> y <p>:

  \vspace{\baselineskip}
  \begin{lstlisting}
<h1>Mi primer encabezado</h1>
<p>Mi primer párrafo.</p>
  \end{lstlisting}

  \vspace{\baselineskip}
  El elemento <h1> define un encabezado.
  Tiene una etiqueta de inicio <h1> y una etiqueta de fin </h1>.

  \vspace{\baselineskip}
  El elemento <p> define un párrafo.
  Tiene una etiqueta de inicio <p> y una etiqueta de fin </p>:

\end{frame}

\begin{frame}[fragile]
  \frametitle{Nunca te saltes la etiqueta final}

  Algunos elementos HTML se mostrarán correctamente,
  incluso si se olvida la etiqueta final:

  \vspace{\baselineskip}
  \begin{lstlisting}
<html>
<body>

<p>Estes es un párrafo
<p>Estes es un párrafo

</body>
</html>
  \end{lstlisting}

  \begin{alertblock}{Nota:}
    Sin embargo, ¡nunca confíes en esto!,
    ¡Pueden ocurrir resultados inesperados y errores si
    se olvida la etiqueta final!
  \end{alertblock}
\end{frame}

\begin{frame}[fragile]
  \frametitle{Elementos HTML vacíos}

  Los elementos HTML sin contenido se denominan elementos vacíos.

  La etiqueta <br> define un salto de línea y es un elemento vacío
  sin una etiqueta de cierre:

  \vspace{\baselineskip}
  \begin{lstlisting}
<p>Esto es un <br> párrafo con un salto de línea.</p>
  \end{lstlisting}
\end{frame}

\begin{frame}[c]{HTML no distingue entre mayúsculas y minúsculas}

  Las etiquetas HTML no distinguen entre mayúsculas y minúsculas:
  <P> significa lo mismo que <p>.

  \vspace{\baselineskip}
  El estándar HTML no requiere etiquetas en minúsculas, pero W3C
  (World Wide Web Consortium)
  \textbf{recomienda} minúsculas en HTML y \textbf{exige} minúsculas
  para tipos de documentos más estrictos como XHTML.

  \begin{exampleblock}{Nota:}
    En este curso siempre usamos nombres de etiquetas en minúsculas.
  \end{exampleblock}
\end{frame}

\section{Atributos HTML}

\begin{frame}[c]{Atributos HTML}
  Los atributos HTML proporcionan información adicional
  sobre los elementos HTML.
  \begin{itemize}
    \item Todos los elementos HTML pueden tener \textbf{atributos}
    \item Los atributos proporcionan \textbf{información adicional}
      sobre los elementos .
    \item Los atributos siempre se especifican en la \textbf{etiqueta
      de inicio}
    \item Los atributos generalmente vienen en pares de nombre/valor
      como: \textbf{nombre="valor"}
  \end{itemize}
\end{frame}

\begin{frame}[fragile]
  \frametitle{El atributo href}

  La etiqueta <a> define un hipervínculo.
  El atributo \textbf{href} especifica la URL de la página a la
  que va el enlace:

  \vspace{\baselineskip}
  \begin{lstlisting}
 <a href="https://www.w3schools.com">Visita W3Schools</a>
  \end{lstlisting}
\end{frame}

\begin{frame}[fragile]
  \frametitle{El atributo src}

  La etiqueta <img> se utiliza para incrustar una imagen en una
  página HTML. El atributo \textbf{src} especifica la ruta a la
  imagen que se va a mostrar:

  \vspace{\baselineskip}
  \begin{lstlisting}
 <img src="nombre_de_la_imagen.jpg">
  \end{lstlisting}
\end{frame}

\begin{frame}[c]{URL absolutas y relativas}

  Hay dos formas de especificar la URL en el src atributo:

  \vspace{\baselineskip}
  \begin{description}
    \item[URL absoluta] enlaces a una imagen externa alojada en
      otro sitio web. Ejemplo:
      src="https://www.w3schools.com/images/img\_girl.jpg".
  \end{description}

  \begin{alertblock}{Nota:}
    Las imágenes externas pueden estar protegidas por derechos de autor.
    Si no obtiene permiso para usarlo, es posible que esté violando las
    leyes de derechos de autor. Además, no puede controlar imágenes
    externas; se puede quitar o cambiar repentinamente.
  \end{alertblock}
\end{frame}

\begin{frame}[c]{URL absolutas y relativas}

  \begin{description}
    \item[URL relativa] enlaces a una imagen alojada en el sitio web.
      Aquí, la URL no incluye el nombre de dominio.

      \vspace{\baselineskip}
      Si la URL comienza sin una barra inclinada (una diagonal),
      será relativa a la página actual.
      Ejemplo: src="img\_girl.jpg".

      \vspace{\baselineskip}
      Si la URL comienza con una barra
      inclinada (una diagonal), será relativa al dominio.
      Ejemplo: src="/images/img\_girl.jpg".
  \end{description}

  \begin{exampleblock}{Sugerencia:}
    Casi siempre es mejor usar URL relativas.
    No se romperán si cambias de dominio.
  \end{exampleblock}
\end{frame}

\begin{frame}[fragile]
  \frametitle{Los atributos de ancho y alto}

  La etiqueta <img> también debe contener los atributos width y height,
  que especifican el ancho y el alto de la imagen (en píxeles):

  \vspace{\baselineskip}
  \begin{lstlisting}
<img src="una_imagen.jpg" width="500" height="600"> 
  \end{lstlisting}
\end{frame}

\begin{frame}[fragile]
  \frametitle{El atributo alt}

  El atributo alt requerido para la etiqueta <img> especifica un
  texto alternativo para una imagen, si la imagen por algún motivo no
  se puede mostrar. Esto puede deberse a una conexión lenta, a un error
  en el atributo src, o si el usuario usa un lector de pantalla.

  \vspace{\baselineskip}
  \begin{lstlisting}
<img src="imagen.jpg" alt="Explicación de la imagen"> 
  \end{lstlisting}

  \vspace{\baselineskip}
  El texto alternativo aparecerá si pasan y dejan el ratón sobre la imagen,
  también aparece si la imagen no existe.
\end{frame}

\begin{frame}[fragile]
  \frametitle{El atributo de estilo}

  El atributo style se usa para agregar estilos a un elemento,
  como color, fuente, tamaño y más.

  \vspace{\baselineskip}
  \begin{lstlisting}
<p style="color:red;">Este es un párrafo rojo.</p> 
  \end{lstlisting}
\end{frame}

\begin{frame}[fragile]
  \frametitle{El atributo lang}

  Siempre debe incluir el atributo \textbf{lang} dentro de la
  etiqueta <html>, para declarar el idioma de la página web.
  Esto está destinado a ayudar a los motores de búsqueda y
  navegadores.

  El siguiente ejemplo especifica el inglés como idioma:

  \vspace{\baselineskip}
  \begin{lstlisting}
<!DOCTYPE html>
<html lang="en">
<body>
...
</body>
</html>
  \end{lstlisting}
\end{frame}

\begin{frame}[fragile]
  \frametitle{El atributo lang}

  Los códigos de países también se pueden agregar al código de idioma
  en el atributo lang. Entonces, los dos primeros caracteres definen
  el idioma de la página HTML y los dos últimos caracteres definen el
  país.

  El siguiente ejemplo especifica inglés como idioma y Estados Unidos
  como país:

  \vspace{\baselineskip}
  \begin{lstlisting}
<!DOCTYPE html>
<html lang="en-US">
<body>
...
</body>
</html>
  \end{lstlisting}
\end{frame}
